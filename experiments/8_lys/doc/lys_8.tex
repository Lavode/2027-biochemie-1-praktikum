\documentclass[a4paper,german]{scrreprt}

% Uncomment to optimize for double-sided printing.
% \KOMAoptions{twoside}

% Set binding correction manually, if known.
% \KOMAoptions{BCOR=2cm}

% Localization options
\usepackage[german]{babel}
\usepackage[T1]{fontenc}
\usepackage[utf8]{inputenc}

% Enhanced verbatim sections. We're mainly interested in
% \verbatiminput though.
\usepackage{verbatim}

% PDF-compatible landscape mode.
% Makes PDF viewers show the page rotated by 90°.
\usepackage{pdflscape}

% Advanced tables
\usepackage{tabu}
\usepackage{longtable}

% Fancy tablerules
\usepackage{booktabs}

% Graphics
\usepackage{graphicx}

% Current time
\usepackage[useregional=numeric]{datetime2}

% Float barriers.
% Automatically add a FloatBarrier to each \section
\usepackage[section]{placeins}

% Custom header and footer
\usepackage{fancyhdr}

\usepackage{geometry}
\usepackage{layout}

% Math tools
\usepackage{mathtools}
% Math symbols
\usepackage{amsmath,amsfonts,amssymb}
\usepackage{amsthm}

% SI units
\usepackage{siunitx}
\DeclareSIUnit\Molar{\textsc{m}}

% Chemistry
\usepackage{mhchem}

\DeclarePairedDelimiter\abs{\lvert}{\rvert}

\pagestyle{plain}
% \fancyhf{}
% \lhead{}
% \lfoot{}
% \rfoot{}
% 
% Source code & highlighting
\usepackage{listings}

% Convenience commands
\newcommand{\mailsubject}{2027 - Praktikum Biochemie 1}
\newcommand{\maillink}[1]{\href{mailto:#1?subject=\mailsubject}
                               {#1}}

% Should use this command wherever the print date is mentioned.
\newcommand{\printdate}{\today}

\subject{2027 - Praktikum Biochemie 1}
\title{8 - Renaturierung von Hühnereiweiss-Lysozym}

\author{David Willimann \maillink{david.willimann@students.unibe.ch} - 17-118-977 \\ Michael Senn \maillink{michael.senn@students.unibe.ch} - 16-126-880}

\date{\printdate}

% Needs to be the last command in the preamble, for one reason or
% another. 
\usepackage{hyperref}


\begin{document}
\maketitle

\chapter{Einleitung \& Theorie}

\section{Einleitung}

In diesem Experiment wurde die reversible thermische und chemische
Denaturierung von Hühnereiweiss-Lysozym untersucht. Hierzu wurde einem Ei
Eiweiss entnommen, und die enthaltenen Proteine chemisch und thermisch
denaturiert. Nach erfolgter Renaturierung wurde mittels Photometer die
enzymatische Aktivität gegenüber Zellwänden gemessen, und mit nativem Lysozym
verglichen.

\section{Theorie}

\subsection{Effekt von 6M Harnstoff auf Lysozym}

Harnstoff, als chaotropes Salz, führt zu einer Störung der Wasserstoffbrücken
was - durch Wegfall des hydrophoben Effekts - zur Denaturierung von Lysozym
führt. Da der hydrophobe Effekt jedoch wegfällt, wird das denaturierte Protein
in Lösung übergehen und nicht ausfällen.

\subsection{Effekt der Erwärmung von Lysozym auf \SI[detect-weight]{90}{\celsius}}

Die Erhöhung der kinetischen Energie führt zu einer thermischen Denaturierung
von Lysozym. Durch den weiterhin vorhandenen hydrophoben Effekt wird das
Protein allerdings ausfällen.

Zu beachten ist, dass die kovalenten Disulphidbrücken erhalten bleiben.

\subsection{Effekt der Zugabe von \SI[detect-weight]{0.01}{\Molar} \ce{HCl} zu Lysozym}

Der pH einer 0.01 M \ce{HCl}-Lösung ist gegeben als $\texttt{pH} \approx
-\log_{10}(0.01) = 2$. Solch eine Lösung wird somit jegliche protonierbaren
Seitenketten protonieren. Diese Ladungsverschiebung führt zu einem Wegfall der
ionischen Wechselwirkungen zwischen den Seitenketten, was zu einer
Denaturierung des Lysozyms führt. Da der hydrophobe Effekt weiterhin vorhanden
ist, wird es zu einer Ausfällung des Proteins kommen.

\subsection{Reaktionsmechanismus von Lysozym}

\subsection{Theoretisch mögliche Disulphid-Brücken in Lysozym}

Lysozym hat vier Disulphid-Brücken, und damit acht Cystein-Reste. Damit sind
für die erste Bindung $\binom{8}{2}$, für die zweite $\binom{6}{2}$ etc
Möglichkeiten vorhanden. Da die Reihenfolge der Bindungen keine Rolle spielt,
folgt die totale Anzahl Möglichkeiten als:

\[
	\frac{\binom{8}{2} \binom{6}{2} \binom{4}{2} \binom{2}{2}}{\frac{n}{2}!} = 105
\]

Dies bedeutet, dass Faltungsintermediate so aufgebaut sein müssen, dass die
Bildung der korrekten Disulfid-Brücken bevorzugt wird.

\chapter{Material \& Methoden}

\section{Material}

\begin{itemize}
\item 2 \SI{500}{ml} Bechergläser
\item 1 \SI{100}{ml} Messzylinder
\item 1 Styroporbox, gefüllt mit Eis
\item Dialyseklammern
\item Pipetten \SIrange{10}{100}{\ul}, \SIrange{100}{1000}{\ul}
\item Spritze mit Nadel
\item Magnetrührer
\item Persitaltische Pumpe
\end{itemize}

\section{Chemikalien und Lösungen}

\begin{itemize}
	\item Tris(hexdroxymethyl)amonimethan
	\item EDTA (Triplex III)
	\item Zellwände von Micrococcus
	\item Di-Kaliumhydrogenphosphat
	\item Kalium-dihydrogenphosphat
	\item Guanidine-Hydrochlorid
	\item 1,4 Dithioerythrit (DTE)
	\item Glutathion, oxidiert (GSSG)
	\item Glutathion, reduziert (GSH)
	\item Arginine

	\item 10x TE (Tris-EDTA) \SI{1}{\Molar} Tris
	\begin{itemize}
		\item \SI{1}{\Molar} Tris
		\item \SI{10}{\milli\Molar} Tris
		\item pH: 8.5
	\end{itemize}

	\item Assay-Lösung
	\begin{itemize}
		\item \SI{30}{\milli\Molar} Kaliumphosphat
		\item \SI{0.1}{\percent} (w/v) BSA
		\item \SI{10}{mg} Zellwände Micrococcus lysodeiktikus pro \SI{2.5}{ml}
		\item pH: 7.4
	\end{itemize}

	\item Guanidine-\ce{HCl}
	\begin{itemize}
		\item \SI{8}{\Molar} Guanidine-hydrochlorid
		\item \SI{0.1}{\Molar} Tris
		\item \SI{1}{\milli\Molar} EDTA
	\end{itemize}
	\item GSSG-Puffer
	\begin{itemize}
		\item \SI{0.1}{\Molar} Tris
		\item \SI{1}{\milli\Molar} EDTA
		\item \SI{1}{\milli\Molar} Glutathion, oxidiert (GSSG)
		\item \SI{0.1}{\milli\Molar} Glutathion, reduziert (GSH)
		\item \SI{0.8}{\milli\Molar} Arginine
		\item pH: 8.5
	\end{itemize}
\end{itemize}



\chapter{Resultat}

\chapter{Fazit}

\end{document}
