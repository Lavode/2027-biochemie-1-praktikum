\documentclass[a4paper,english]{scrreprt}

% Uncomment to optimize for double-sided printing.
% \KOMAoptions{twoside}

% Set binding correction manually, if known.
% \KOMAoptions{BCOR=2cm}

% Localization options
\usepackage[german]{babel}
\usepackage[T1]{fontenc}
\usepackage[utf8]{inputenc}

% Enhanced verbatim sections. We're mainly interested in
% \verbatiminput though.
\usepackage{verbatim}

% PDF-compatible landscape mode.
% Makes PDF viewers show the page rotated by 90°.
\usepackage{pdflscape}

% Advanced tables
\usepackage{tabu}
\usepackage{longtable}
\usepackage{dcolumn}
\newcolumntype{d}[1]{D{.}{\cdot}{#1} }

% Fancy tablerules
\usepackage{booktabs}

% Graphics
\usepackage{graphicx}

% Current time
\usepackage[useregional=numeric]{datetime2}

% Float barriers.
% Automatically add a FloatBarrier to each \section
\usepackage[section]{placeins}

% Custom header and footer
\usepackage{fancyhdr}

\usepackage{geometry}
\usepackage{layout}

% Math tools
\usepackage{mathtools}
% Math symbols
\usepackage{amsmath,amsfonts,amssymb}
\usepackage{amsthm}

% SI units
\usepackage{siunitx}
\DeclareSIUnit\Molar{\textsc{m}}
\DeclareSIUnit\mole{mol}
\DeclareSIUnit\rpm{rpm}
\DeclareSIUnit\cfu{cfu}

% Chemistry
\usepackage{mhchem}

% Subfigures & captions
\usepackage{subcaption}
\usepackage{wrapfig}

\DeclarePairedDelimiter\abs{\lvert}{\rvert}

\pagestyle{plain}
% \fancyhf{}
% \lhead{}
% \lfoot{}
% \rfoot{}
% 
% Source code & highlighting
\usepackage{listings}

% Convenience commands
\newcommand{\mailsubject}{2027 - Lab course biochemistry 1}
\newcommand{\maillink}[1]{\href{mailto:#1?subject=\mailsubject}
                               {#1}}

% Should use this command wherever the print date is mentioned.
\newcommand{\printdate}{\today}

\subject{2027 - Lab course biochemistry 1}

\title{11 - Isolierung und Charakterisierung von Membranlipiden}

\author{Michael Senn \maillink{michael.senn@students.unibe.ch} - 16-126-880 - Gruppe 14}

\date{\printdate}

% Needs to be the last command in the preamble, for one reason or
% another. 
\usepackage{hyperref}


\begin{document}
\maketitle

\chapter{Einleitung}

Zwecks Kompartmentalisierung zwischen wie auch innerhalb der Zelle finden sich
in vielen Organismen Membrane. Diese sind selektiv durchlässig für verschiedene
metabolische Produkte, und dienen dem kontrollierten Informationsaustausch
zwischen Zellen. Ein wichtiger Bestandteil dieser Membrane sind Membranlipide.
Eine deren Klasse - die Phospholipide - wurde in diesem Versuch genauer
untersucht.

\section{Phospholipide}

Phospholipide bestehen im Allgemeinen aus zwei hydrophoben Fettsäuren, und
einem hydrophilen Kopf bestehend aus einer Phosphat-Gruppe, verbunden mittels
einem Glycerol. Dies erlaubt in wässrigem Medium die Bildung einer
Doppellipidschicht, in welcher die Fettsäuren gegen innen, und die
Phosphat-Gruppen gegen aussen zeigen.

Durch Anlagerung verschiedener organischer Moleküle an der Phosphatgruppe
ergeben sich Klassen von Phospholipiden mit unterschiedlich polaren
Kopfgruppen.

\subsection{Polarität der Kopfgruppen}

Für vier dieser Klassen soll hier die Polarität der Kopfgruppe hergeleitet
werden. Dies sind Phosphatidylserine (PS), Phosphatidylcholine (PC),
Phosphatidylethanolamine (PE), und Phosphatidylinositol (PI).

\begin{figure}
	\centering
	\includegraphics[width=0.9\textwidth]{img/phospholipide.png}
	\caption{Struktur von vier Klassen von Phospholipiden, aus \cite{handoutv11}}
	\label{fig:phospholipide}
\end{figure}

Phosphatidylcholine mit einem tertiären Amin besitzt die polarste Kopfgruppe
dieser vier Klassen, da eine Verminderung der Ladung aufgrund der stabilen
\ce{CH3} Reste kaum möglich ist.

Phosphatidylethanolamin hat Aufgrund der in Abbildung \ref{fig:phospholipide}
ersichtlichen positiv geladenen Amino-Gruppe die am zweitstärksten polare
Kopfgruppe. Da es sich hierbei aber um ein primäres Amin handelt, kann diese
Ladung durch Deprotonierung neutralisiert werden.

Phosphatidylserin hat, in dem im Versuch verwendeten Lösungsmittel Chloroform,
nur eine schwache Ladung auf der Carboxylgruppe, und somit den drittpolarsten
Kopf.

Phosphatidylcholine besitzt ein, in Abbildung \ref{fig:phospholipide}
sichtbares, Ringsystem. Dessen elektronenstabilisiernde Eigenschaft führt dazu,
dass PC die am wenigsten polare Kopfgruppe besitzt.

\subsection{Einfluss der Fettsäuren auf auf Membranrigidität}

Der Aufbau der Fettsäuren beinflusst die Rigidität der Membrane auf zwei Arten.
Längere Fettsäuren führen, aufgrund stärkerer Van-der-Waals-Kräfte, zu einer
rigideren Membran. Ungesättigte Fettsäuren - das heisst Fettsäuren mit C-C
Doppelbindungen - besitzten einen Knick welcher dazu führt dass die Fettsäuren
weniger dicht gepackt sind, was zu einer weniger rigiden Membran führt

\section{Chromatographie}

Die Chromatographie nutzt Interaktionen zwischen Molekülen dazu, ein Gemisch
basierend auf einer oder mehreren Eigenschaften aufzutrennen. Allen
Chromatographiemethoden ist gemein dass zwei Phasen verwendet werden - die
mobile Phase in welcher das aufzutrennende Gemisch wandert, und die stationäre
Phase, deren Wechselwirkungen mit dem Gemisch bestimmen wie schnell die
einzelnen Komponenten wandern.

Je nach Aufbau der stationären Phase kann beispielsweise nach Ladung oder
Grösse aufgetrennt werden. In unserem Versuch wurden zwei Arten der
Chromatographie verwendet - die Dünnschichtchromatographie und die
Gaschromatographie.

\subsection{Dünnschichtchromatographie}

In der Dünnschichtchromatographie wurde ein apolares Lösungsmittel als mobile
Phase, und ein polares Silicagel als stationäre Phase verwendet. Phospholipide
mit polaren Kopfgruppen interagierten stärker mit der stationären Phase als
jene mit apolaren Kopfgruppen, wodurch eine Auftrennung nach Polarität der
Kopfgruppe stattfand - je apolarer die Kopfgruppe, umso weiter wanderten die
Lipide auf der stationären Phase.

Zwecks Analyse der auftegrennten Phospholipide wurden die Versuchsreihen mit
Farbstoffen eingefärbt. Dichlorofluoreszin zwecks Detektion von Lipiden,
Ninhydrin zwecks Detektion von freinen Aminen, und Molbydän zwecks Detektion
von Phosphatgruppen.

\subsection{Gaschromatographie}

In der Gaschromatographie wurde als stationäre Phase ein inertes Gas verwendet,
welches durch eine Säule - ein langer dünner Schlauch - floss. Als stationäre
Phase diente Quarzglas, mit welchem das Innere der Säule ausgekleidet war.

Die aufzutrennenden Phospholipide wurden durch langsames erhöhen der Temperatur
in die Gasphase überührt, wobei jene mit polaren Kopfgruppen aufgrund stärkerer
Interaktion mit der stationären Phase erst bei höheren Temperaturen in die
Gasphase übergingen.

Zur Detektion der Moleküle in der Gasphase diente ein
Flammenionisationsdetektor am Ende der Säule, zusammen mit einer
bereitgestellten Messreihe einer Referenzlösung von verschiedenen
Phospholipiden.

\section{Transesterifizierung}

Da die Gaschromatographie nur mit flüchtigen Stoffen funktioniert, wurden die
Phospholipide vorgängig mittels Transesterifizierung zu Fettsäuremethylestern
überführt. \cite{handoutv11}

Hierbei wird durch Zugabe eines Überschusses an Methanolat, und Heizen auf
\SI{75}{\celsius} für \SI{7}{\min} die endotherme Reaktion in Richtung der
Fettsäuremethylester bevorzugt. Die Reaktion wird durch Zugabe von \ce{KH2PO4}
abgebrochen, welches die überschüssigen freien Methanolate protoniert.

Der Reaktionsmechanismums ist in Abbildung \ref{fig:transesterifizierung}
zusammengefasst.

\begin{figure}
	\centering
	\includegraphics[width=0.3\textwidth]{img/transesterifizierung.png}
	\caption{Transesterifizierung von Phospholipden, aus \cite{handoutv11}}
	\label{fig:transesterifizierung}
\end{figure}

Durch Zugabe von Hexan lassen sich die vergleichsweise apolaren
Fettsäuremethyleser extrahieren und für die Gaschromatographie verwenden.

\section{Ziel}

Ziel des Experimentes war es, die Zusammensetzung der Lipiden in E. coli
Membranen bei unterschiedlichen Temperaturen zu untersuchen. Hierzu wurden
Lipide aus E. coli Kulturen welche bei \SI{20}{\celsius} respektive
\SI{37}{\celsius} gezüchtet wurden extrahiert, und mittels Chromatographie
aufgetrennt und analysiert.

\subsection{Hypothesen}

% TODO: Drei?

Zwei Hypothesen wurden aufgestellt:
\begin{enumerate}
	\item In der \SI{20}{\celsius} Kultur sind Fettsäuren durchschnittlich
		kürzer als in der \SI{37}{\celsius} Kultur
	\item In der \SI{20}{\celsius} Kultur hat es mehr ungesättgte
		Fettsäuren als in der \SI{37}{\celsius} Kultur
\end{enumerate}

\chapter{Durchführung}

% TODO: Verweis auf Protokoll, Messwerte & Beobachtungen, potentielle Abweichungen vom Protokoll

Der Versuch wurde gemäss bereitgestelltem Protokoll \cite{skriptv11}
durchgeführt. Abweichungen, Beobachtungen und Messwerte sind hier dokumentiert.

\section{Massebestimmung der isolierten Lipide}

\subsection{Isolation der Membranlipide}

Bei Zugabe des heisen Methanols bildete sich ein Ausfall. Dieser Ausfall
enthielt alle Zellbestandteile welche nicht im apolaren Methanol lösbar waren -
beispielsweise polare Protein, oder die negativ geladene DNA.

Nach der Abdestillation des Methanols wurden die Lipidrückstände versehentlich
mit reinem \ce{CH3Cl} anstelle einer 1:1 Mischung aus \ce{CH3Cl} und \ce{MeOH}
gelöst und durch den Baumwollfilter gelassen. Zur Korrektur wurden die
Rückstände im Rundkolben ein weiteres Mal mit \SI{1}{\ml} 1:1
\ce{CH3Cl}:\ce{MeOH} gelöst, und diese Lösung ebenfalls durch den
Baumwollfilter gegeben. Das zweite Nachspülen mit 1:1 \ce{CH3Cl}:\ce{MeOH} fand
gemäss Protokoll statt.

Bei der Zugabe der Lösungsmittels wurde eine leichte Verfärbung beobachtet, was
impliziert dass sich gewisse Bestandteile des Rückstandes lösten.

\subsection{Bestimmung der isolierten Lipidmenge}

Da die Analysewage defekt war, musste die Messung mit der Oberschalenwage
durchgeführt werden. Aufgrund deren fehlenden Genauigkeit konnte die isolierte
Menge nicht bestimmt werden, und die Menge ans isolierten Lipiden wurde von
Auge geschätzt. Alle unsere Proben wurden in \SI{1}{\ml} \ce{CHCl3} gelöst, die
gewünschte Endkonzentration von \SI{20}{\mg\per\ml} war damit nur approximativ
möglich.

\section{Dünnschichtchromatographie}

Die DC Platte wurde gemäss Ladeschema in Tabelle \ref{tbl:dc_ladeschema} mit
den extrahierten Lipiden der beiden Kulturen, und einer Referenzlösung,
bestückt.

\begin{table}
	\centering
	\begin{tabu}{l|ccc|c|ccc|c|ccc|c}
		\toprule
		                  & 1 & 2 & 3 & 4 & 5 & 6 & 7 & 8 & 9 & 10 & 11 & 12 \\
		\midrule
		\SI{20}{\celsius} & x &   &   &   & x &   &   &   & x &    &    &    \\
		\SI{37}{\celsius} &   & x &   &   &   & x &   &   &   & x  &    &    \\
		Referenz          &   &   & x &   &   &   & x &   &   &    & x  &    \\
		\midrule
		Farbstoff         & \multicolumn{3}{c|}{Dichlorof.} & & \multicolumn{3}{c|}{Ninhydrin} & & \multicolumn{3}{c|}{Molybdän} \\
		\bottomrule
	\end{tabu}
	\caption{Ladeschema für Dünnschichtchromatographie}
	\label{tbl:dc_ladeschema}
\end{table}




\bibliographystyle{vancouver}
\bibliography{references}

\end{document}
