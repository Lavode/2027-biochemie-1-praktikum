\documentclass[a4paper,german]{scrreprt}

% Uncomment to optimize for double-sided printing.
% \KOMAoptions{twoside}

% Set binding correction manually, if known.
% \KOMAoptions{BCOR=2cm}

% Localization options
\usepackage[german]{babel}
\usepackage[T1]{fontenc}
\usepackage[utf8]{inputenc}

% Enhanced verbatim sections. We're mainly interested in
% \verbatiminput though.
\usepackage{verbatim}

% PDF-compatible landscape mode.
% Makes PDF viewers show the page rotated by 90°.
\usepackage{pdflscape}

% Advanced tables
\usepackage{tabu}
\usepackage{longtable}
\usepackage{dcolumn}
\newcolumntype{d}[1]{D{.}{\cdot}{#1} }

% Fancy tablerules
\usepackage{booktabs}

% Graphics
\usepackage{graphicx}

% Current time
\usepackage[useregional=numeric]{datetime2}

% Float barriers.
% Automatically add a FloatBarrier to each \section
\usepackage[section]{placeins}

% Custom header and footer
\usepackage{fancyhdr}

\usepackage{geometry}
\usepackage{layout}

% Math tools
\usepackage{mathtools}
% Math symbols
\usepackage{amsmath,amsfonts,amssymb}
\usepackage{amsthm}

% SI units
\usepackage{siunitx}
\DeclareSIUnit\Molar{\textsc{m}}

% Chemistry
\usepackage{mhchem}

% Subfigures & captions
\usepackage{subcaption}

\DeclarePairedDelimiter\abs{\lvert}{\rvert}

\pagestyle{plain}
% \fancyhf{}
% \lhead{}
% \lfoot{}
% \rfoot{}
% 
% Source code & highlighting
\usepackage{listings}

% Convenience commands
\newcommand{\mailsubject}{2027 - Praktikum Biochemie 1}
\newcommand{\maillink}[1]{\href{mailto:#1?subject=\mailsubject}
                               {#1}}

% Should use this command wherever the print date is mentioned.
\newcommand{\printdate}{\today}

\subject{2027 - Praktikum Biochemie 1}
\title{7 - Proteinimport in Hefe Mitochondrien}

\author{Michael Senn \maillink{michael.senn@students.unibe.ch} - 16-126-880}

\date{\printdate}

% Needs to be the last command in the preamble, for one reason or
% another. 
\usepackage{hyperref}


\begin{document}
\maketitle

\chapter{Einleitung}

\section{Zielsetzung}

Ziel des Experiments war es, den Transport von Proteinen durch die
Mitochondrien-Matrix zu untersuchen. Hierzu wurde eine mit
\ce{^{35}S-\text{Methionin}} markierte Alkoholdehydrogenase 3 (ADHIII) in
Mitochondrien importiert, und deren Import im Anschluss durch
SDS-Polyacrylamidgelelektrophorese analysiert.

\section{Protein-Import in Mitochondrien}

Viele der Proteine die im Mitochondrium benötigt werden, sind im Zellkern der
Zelle codiert, werden in deren Cytoplasma synthetisiert, und anschliessend als
komplettes Protein in das Mitochondrium importiert.

Ausgelöst wird dies durch eine positiv geladene N-terminale Sequenz der
jeweiligen Proteine, die an Rezeptor-Proteine in der Membran binden. Im
Anschluss werden die Proteine durch den TOM-Komplex in den Zellzwischenraum,
und darauf folgend durch den TIM-Komplex in das Zellinnere
geschleust\cite{protein_transport}.

Getrieben wird dieser Transport Matrix-seitig durch mHsp 70, das
mitochondrielle Heat-Shock Protein 70, unter Verbrauch von ATP.

In den meisten Fällen wird die N-terminale Sequenz, die als Signal für den
Transport diente, im Zellinneren der Mitochondrien abgetrennt - das Protein
someit einige Sequenzen kürzer.

\section{SDS Page}

SDS-Page ist eine Methode der Elektrophorese, welche erlaubt, Proteine der
Grösse nach zu ordnen. Hierzu werden die Proteine mittels SDS denaturiert,
wobei sich das negative geladene SDS an das dentaurierte Protein anlagert, und
so zu einer uniformen Ladungsverteilung führt. Dies stellt sicher, dass die
Proteine in folgenden Schritten nur gemäss ihrer Grösse, und nicht gemäss
potentieller Ladungsunterschiede, angeordnet werden.

Diese Proteine werden dann in ein Gel - ein Polyacrylamid, hergestellt aus der
Polymerisation von Acrylamid - gegeben. Durch Anlegen eines elektrischen Feldes
wandern die Proteine durch das Gel, kleinere Proteine schneller - und damit
weiter - wandern als grosse.

Zum Abschluss werden die Proteine fixiert und eingefärbt, was damit zu einer
sichtbaren Aufteilung der Grösse nach führt.

\chapter{Chemikalien}

\section{Proteinase K}

Proteinase K ist eine Proteinase mit tiefer Spezifität - sie spaltet Proteine
bevorzugt nach hydrophoben Aminosäuren, führt aber bei langer Inkubationszeit
auch zu einem kompletten Abbau des Proteins. Durch das Vorhandensein von
denaturierenden Reagenzien wie SDS lässt sich die Aktivität zusätzlich
steigern, da das Protein für die Proteinase einfacher zugänglich wird. \cite{stimulation_proteinase}

Im Experiment diente Proteinase K dazu, das Substrat das ausserhalb der Zelle
verblieb abzubauen, bevor die Mitochondrien aufgebrochen wurden. Um zu
verhindern dass es die Zellmembran der Mitochondrien angriff, wurde sowohl die
Inkubationszeit, als auch die Inkubationstemperatur, tief gehalten.

\section{PMSF}

Phenylmethylsufonyl-Fluorid (PMSF) ist ein Serin-Proteasen-Inhibitor, der somit
Eigenschaften eines Nervengiftes hat.

Im Experiment wurde mittels PMSF die vorher zugegebene Proteinkinase
inaktiviert. Dies verhinderte, dass nach Aufbrechen der Mitochondrien die
Kinase die Proteine des Mitochondrium-Inneren abbauten.

\section{AVO}

AVO ist eine Mischung von drei wirkstoffen mit antibakterieller Wirkung -
Antimycin A, Valinomycin, und Oligomycin.

Antimycin A Bindent an den Komplex 3 und verhindert den Transfer von Elektronen
als Teil der oxidativen Phosphorylierung. Dies verhindert den Aufbau des
Protonengradienten, und somit schlussendliche die Synthese von ATP.

Valinomycin fungiert als Transporter für Kalium, der in der Lage ist, durch
Lipiddoppelschichten zu diffundieren. Damit führt es zu einem Transport von
Kalium aus dem Zellinneren in den extrazellulären Raum, was das
Membranpotential vermindert.

Oligomycin blockiert den Protonen-Kanal der ATP-Synthase, und verhindert so
direkt, dass ATP synthetisiert werden kann.

Im Experiment diente diese Mischung dazu, das Membranpotential abzubauen, und den
Import von Proteinen - der sowohl ein Membranpotential als auch ATP benötigt - zu
stoppen.

\section{NADH}

NADH ist ein universeller Elektronen-Träger, der in vielen biochemischen Pfaden
eine wichtige Rolle spielt.

Im Experiment erlaubte NADH den Mitochondrien, ihr Membranpotential aufzubauen
und aufrechtzuerhalten.

\section{BSA \& Sucrose}

Bovines Serumalbumin (BSA) ist ein Protein aus dem Blutplasma von Kühen, mit
hoher und breiter Bindungsfähigkeit für polare Stoffe wie Wasser oder
Metallionen, wie auch weniger polare Stoffe wie Fettsäuren.

Im Experiment diente es dazu, sowohl ADHIII als auch die Mitochondrien zu
binden, um sie während beispielsweise des Pipettiervorganges vor mechanischen
Einwirkungen zu schützen.

Sucrose - ein einfacher Zweifachzucker bestehend aus einer Glucose und einer
Fructose verbunden via eine Ether-Bindung, stabilisierte die Mitochondrien
durch Bindung an deren Membran.

\section{Kreatinkinase \& Phosphokreatin}

Phosphokreatin ist ein phosphoryliertes Kreatin-Molekül, das als Energiereserve
in beispielsweise Muskelzellen und dem Gehirn dient. Dies im Zusammenspiel mit
der Kreatinkinase, die eine Phosphoryl-Gruppe des Phosphokreatins auf ADP
überträgt, und somit ATP regeneriert.

Im Experiment stellte die zugegebene Reserve von Phosphokreatin und
Kreatinkinase sicher, dass für den Proteinimport genügend ATP zur Verfügung
stand, ohne dass grosse Mengen ATP zugegeben werden musste.

\section{SDS}

Natriumdodecylsulfat (SDS, vom Englischen sodium dodecyl sulfate) ist eine
organische Verbindung mit denaturierenden Eigenschaften. Durch die Kombination
aus polarem Kopf und apolarem Schwanz findet es auch Verwendung in
Reinigungsmitteln, da es erlaubt, apolare Substanzen in beispielsweise Wasser
zu lösen.

Im Experiment wurden die Proteine mittels SDS denaturiert. Durch die so
entstehende uniforme Ladungsverteilung wurde auch sichergestellt, dass die
Proteine entsprechend ihrer Grösse - ohne Einfluss ihrer Ladung - angeordnet
wurden.

\chapter{Durchführug}

\chapter{Resultate}

\bibliographystyle{plain}
\bibliography{references}

\end{document}
