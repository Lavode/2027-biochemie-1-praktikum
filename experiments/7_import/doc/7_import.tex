\documentclass[a4paper,german]{scrreprt}

% Uncomment to optimize for double-sided printing.
% \KOMAoptions{twoside}

% Set binding correction manually, if known.
% \KOMAoptions{BCOR=2cm}

% Localization options
\usepackage[german]{babel}
\usepackage[T1]{fontenc}
\usepackage[utf8]{inputenc}

% Enhanced verbatim sections. We're mainly interested in
% \verbatiminput though.
\usepackage{verbatim}

% PDF-compatible landscape mode.
% Makes PDF viewers show the page rotated by 90°.
\usepackage{pdflscape}

% Advanced tables
\usepackage{tabu}
\usepackage{longtable}
\usepackage{dcolumn}
\newcolumntype{d}[1]{D{.}{\cdot}{#1} }

% Fancy tablerules
\usepackage{booktabs}

% Graphics
\usepackage{graphicx}

% Current time
\usepackage[useregional=numeric]{datetime2}

% Float barriers.
% Automatically add a FloatBarrier to each \section
\usepackage[section]{placeins}

% Custom header and footer
\usepackage{fancyhdr}

\usepackage{geometry}
\usepackage{layout}

% Math tools
\usepackage{mathtools}
% Math symbols
\usepackage{amsmath,amsfonts,amssymb}
\usepackage{amsthm}

% SI units
\usepackage{siunitx}
\DeclareSIUnit\Molar{\textsc{m}}

% Chemistry
\usepackage{mhchem}

% Subfigures & captions
\usepackage{subcaption}

\DeclarePairedDelimiter\abs{\lvert}{\rvert}

\pagestyle{plain}
% \fancyhf{}
% \lhead{}
% \lfoot{}
% \rfoot{}
% 
% Source code & highlighting
\usepackage{listings}

% Convenience commands
\newcommand{\mailsubject}{2027 - Praktikum Biochemie 1}
\newcommand{\maillink}[1]{\href{mailto:#1?subject=\mailsubject}
                               {#1}}

% Should use this command wherever the print date is mentioned.
\newcommand{\printdate}{\today}

\subject{2027 - Praktikum Biochemie 1}
\title{7 - Proteinimport in Hefe Mitochondrien}

\author{Michael Senn \maillink{michael.senn@students.unibe.ch} - 16-126-880 - Gruppe 14}

\date{\printdate}

% Needs to be the last command in the preamble, for one reason or
% another. 
\usepackage{hyperref}


\begin{document}
\maketitle

\chapter{Einleitung}

\section{Zielsetzung}

Ziel des Experiments war es, den Transport von Proteinen durch die
Mitochondrien-Matrix zu untersuchen. Hierzu wurde eine mit
\ce{^{35}S-\text{Methionin}} markierte Alkoholdehydrogenase 3 (ADHIII) in
Mitochondrien importiert, und deren Import im Anschluss durch
SDS-Polyacrylamidgelelektrophorese analysiert.

\section{Protein-Import in Mitochondrien}

Viele der Proteine die im Mitochondrium benötigt werden, sind im Zellkern der
Zelle codiert, werden in deren Cytoplasma synthetisiert, und anschliessend als
komplettes Protein in das Mitochondrium importiert.

Ausgelöst wird dies durch eine positiv geladene N-terminale Sequenz der
jeweiligen Proteine, die an Rezeptor-Proteine in der Membran binden. Im
Anschluss werden die Proteine durch den TOM-Komplex in den Intermembranraum,
und darauf folgend durch den TIM-Komplex in das Zellinnere
geschleust\cite{protein_transport}.

Getrieben wird dieser Transport Matrix-seitig durch mHsp 70, das
mitochondrielle Heat-Shock Protein 70, unter Verbrauch von ATP.

In den meisten Fällen wird die N-terminale Sequenz, die als Signal für den
Transport diente, im Zellinneren der Mitochondrien abgetrennt - das Protein
somit einige Sequenzen kürzer.

\section{Verwendetes Substrat}

Die \ce{^{35}S-\text{Methionin}} markierte ADHIII wurde in-vitro hergestellt um
sie mit dem radioaktivem Marker zu versehen sowie um sicherzustellen, dass die
N-terminale Sequenz, welche den Import des Proteins auslöst, vorhanden ist.

\section{SDS Page}

SDS-Page ist eine Methode der Elektrophorese welche erlaubt, Proteine der
Grösse nach zu ordnen. Hierzu werden die Proteine mittels SDS denaturiert,
wobei sich das negative geladene SDS an das denaturierte Protein anlagert und
so zu einer uniformen Ladungsverteilung führt. Dies stellt sicher, dass die
Proteine in folgenden Schritten nur gemäss ihrer Grösse, und nicht gemäss
potentieller Ladungsunterschiede, angeordnet werden.

Diese Proteine werden dann in ein poröses Gel - ein Polyacrylamid, hergestellt
aus der Polymerisation von Acrylamid - gegeben. Durch Anlegen eines
elektrischen Feldes wandern die Proteine aufgrund der durch SDS erzeugten
Ladung durch das Gel, wobei kleinere Proteine schneller - und damit weiter -
wandern als grosse.

Zum Abschluss werden die Proteine fixiert und eingefärbt, was die Aufteilung
nach Grösse sichtbar macht.

\chapter{Chemikalien}

\section{Proteinase K}

Proteinase K ist eine Proteinase mit tiefer Spezifität - sie spaltet Proteine
bevorzugt nach hydrophoben Aminosäuren, führt aber bei langer Inkubationszeit
auch zu einem kompletten Abbau des Proteins. Durch das Vorhandensein von
denaturierenden Reagenzien wie SDS lässt sich die Aktivität zusätzlich
steigern, da das Protein für die Proteinase einfacher zugänglich wird. \cite{stimulation_proteinase}

Im Experiment diente Proteinase K dazu, das Substrat das ausserhalb der Zelle
verblieb abzubauen, bevor die Mitochondrien aufgebrochen wurden. Um zu
verhindern dass es die Zellmembran der Mitochondrien angriff, wurde sowohl die
Inkubationszeit, als auch die Inkubationstemperatur, tief gehalten.

\section{PMSF}

Phenylmethylsufonyl-Fluorid (PMSF) ist ein Serin-Proteasen-Inhibitor, der somit
Eigenschaften eines Nervengiftes hat.

Im Experiment wurde mittels PMSF die vorher zugegebene Proteinase
inaktiviert. Dies verhinderte, dass nach Aufbrechen der Mitochondrien die
Kinase die Proteine des Mitochondrium-Inneren abbauten.

\section{AVO}

AVO ist eine Mischung von drei Wirkstoffen mit antibakterieller Wirkung -
Antimycin A, Valinomycin, und Oligomycin.

Antimycin A bindet an den Komplex 3 und verhindert den Transfer von Elektronen
als Teil der oxidativen Phosphorylierung. Dies verhindert den Aufbau des
Protonengradienten, und somit schlussendliche die Synthese von ATP.

Valinomycin fungiert als Transporter für Kalium, der in der Lage ist, durch
Lipiddoppelmembranen zu diffundieren. Damit führt es zu einem Transport von
Kalium aus dem Zellinneren in den extrazellulären Raum, was das
Membranpotential vermindert.

Oligomycin blockiert den Protonen-Kanal der ATP-Synthase und verhindert so
direkt, dass ATP synthetisiert werden kann.

Im Experiment diente diese Mischung dazu, das Membranpotential abzubauen, und den
Import von Proteinen - der sowohl ein Membranpotential als auch ATP benötigt - zu
stoppen.

\section{NADH}

NADH ist ein universeller Elektronen-Träger, der in vielen biochemischen Pfaden
eine wichtige Rolle spielt.

Im Experiment erlaubte NADH den Mitochondrien, ihr Membranpotential aufzubauen
und aufrechtzuerhalten.

\section{BSA \& Sucrose}

Bovines Serumalbumin (BSA) ist ein Protein aus dem Blutplasma von Kühen, mit
hoher und breiter Bindungsfähigkeit für polare Stoffe wie Wasser oder
Metallionen, wie auch weniger polare Stoffe wie Fettsäuren.

Im Experiment diente es dazu, sowohl ADHIII als auch die Mitochondrien zu
binden, um sie während beispielsweise des Pipettiervorganges vor mechanischen
Einwirkungen zu schützen.

Sucrose - ein einfacher Zweifachzucker bestehend aus einer Glucose und einer
Fructose verbunden via eine Ether-Bindung, stabilisierte die Mitochondrien
durch Bindung an deren Membran.

\section{Kreatinkinase \& Phosphokreatin}

Phosphokreatin ist ein phosphoryliertes Kreatin-Molekül, das als Energiereserve
in beispielsweise Muskelzellen und dem Gehirn dient. Dies im Zusammenspiel mit
der Kreatinkinase, die eine Phosphoryl-Gruppe des Phosphokreatins auf ADP
überträgt, und somit ATP regeneriert.

Im Experiment stellte die zugegebene Reserve von Phosphokreatin und
Kreatinkinase sicher, dass für den Proteinimport genügend ATP zur Verfügung
stand, ohne dass grosse Mengen ATP zugegeben werden musste.

\section{SDS}

Natriumdodecylsulfat (SDS, vom Englischen sodium dodecyl sulfate) ist eine
organische Verbindung mit denaturierenden Eigenschaften. Durch die Kombination
aus polarem Kopf und apolarem Schwanz findet es auch Verwendung in
Reinigungsmitteln, da es erlaubt, apolare Substanzen in beispielsweise Wasser
zu lösen.

Im Experiment wurden die Proteine mittels SDS denaturiert. Durch die so
entstehende uniforme Ladungsverteilung wurde auch sichergestellt, dass die
Proteine entsprechend ihrer Grösse - ohne Einfluss ihrer Ladung - angeordnet
wurden.

\chapter{Durchführug}

\section{Herstellung Mastermix}

Der Mastermix wurde hergestellt aus:
\begin{itemize}
	\item \SI{193.75}{\ul} Importpuffer, pH 7.2
	\item \SI{10}{\ul} \SI{0.1}{\Molar} ATP
	\item \SI{5}{\ul} \SI{10}{\mg \per \ml} Kreatinkinase
	\item \SI{5}{\ul} \SI{1}{\Molar} Phosphokreatin
	\item \SI{5}{\ul} \SI{0.1}{\Molar} NADH
\end{itemize}

\section{Vorbereitung Proben}

Vier Proben wurden vorbereitet, gemäss folgender Tabelle.
\\

\begin{tabu}{lcccc}
	\toprule
	Importreaktion & & $-\Delta\Psi$ & +Proteinase K & +Proteinase K, $-\Delta\Psi$ \\
	Probe & 1 & 2 & 3 & 4 \\
	\midrule
	Mastermix & \SI{43.75}{\ul} & \SI{43.75}{\ul} & \SI{43.75}{\ul} & \SI{43.75}{\ul} \\
	AVO Mix & \SI{0}{\ul} & \SI{1}{\ul} &\SI{0}{\ul} & \SI{1}{\ul} \\
	Mitochondrien (\SI{10}{\mg \per \ml}) & \SI{2}{\ul} & \SI{2}{\ul} &\SI{2}{\ul} & \SI{2}{\ul} \\
	Substrat & \SI{2}{\ul} & \SI{2}{\ul} &\SI{2}{\ul} & \SI{2}{\ul} \\
	Proteinase K & \SI{0}{\ul} & \SI{0}{\ul} &\SI{1}{\ul} & \SI{1}{\ul} \\
	\bottomrule
\end{tabu}
\\

Im Detail:
\begin{itemize}
	\item Benötigte Menge des Mastermixes und AVO-Mix zu allen Proben zugegeben, durch Pipettieren gemischt.
	\item Mitochondrien zu allen Proben dazugegeben, durch Antippen gemischt.
	\item Proben für 3 Minuten bei \SI{27}{\celsius} in Heizblock gegeben.
	\item Substrat zu Proben dazugegeben, durch Antippen gemischt.
	\item Alle Proben für 30 Minuten bei \SI{27}{\celsius} inkubiert.
	\item Importreaktion aller Proben durch Zugabe von \SI{0.5}{\ul} AVO gestoppt.
	\item Zugabe \SI{1}{\ul} Proteinase-K zu Proben 3 und 4 zwecks Abbau
		extrazellulären Proteinen.
	\item Inkubation aller Proben für 15 Minuten auf Eis.
	\item Zugabe \SI{0.9}{\ul} PMSF zu allen Proben zwecks Inaktivierung der Proteinase K.
	\item Inkubation aller Proben für 5 Minuten auf Eis.
	\item Zentrifugation der Proben bei 14000 RPM, \SI{4}{\celsius} während 5 Minuten.
	\item Entfernen des Überstandes
\end{itemize}

Zu diesem Zeitpunkt wurden die gebildeten Pellets betrachtet. Es ergaben sich
folgende Beobachtungen und Überlegungen:
\\

\begin{tabu}{lp{2.5cm}p{2.5cm}p{2.5cm}p{2.5cm}}
	\toprule
	Importreaktion & & $-\Delta\Psi$ & +Proteinase K & +Proteinase K, $-\Delta\Psi$ \\
	\midrule
	Probe & 1 & 2 & 3 & 4 \\
	Durchmesser & \SI{2}{mm} & \SI{2}{mm} & \SI{1}{mm} & \SI{1}{mm} \\
	Bestandteile & Mitochondrien, importierte \& extrazelluläre Proteine & Mitochondrien \& extrazelluläre Proteine & Mitochondrien \& importierte Proteine & Mitochondrien \\
	\bottomrule
\end{tabu}
\\

Anschliessend wurde mit den Pellets weitergearbeitet.

\begin{itemize}
	\item Resuspensation der Pellets in \SI{100}{\ul} eiskaltem SEM-Puffer.
	\item Erneute Zentrifugation der Proben bei 14000 RPM, \SI{4}{\celsius} während 5 Minuten.
	\item Entfernen des Überstandes, Resuspensation der Pellets in \SI{30}{\ul} SDS Sample Buffer.
	\item Sofortiges Erhitzen der Problem auf \SI{95}{\celsius} während 5
		Minuten zwecks Aufbrechen der Mitochondrien.

\end{itemize}

\section{Vorbereitung SDS-Gel}

Durch die Assistentin wurde vorgängig das SDS-Gel vorbereitet, basierend auf
folgenden Arbeitsschritten aus dem Praktikumsskript \cite{skriptv7}:

\begin{itemize}
	\item Aufbau der Glasplatten und Spacer wie vorgeführt
	\item Abdichten mit geschmolzener Agarose
	\item Mit \ce{\text{dd}H2O} füllen, um zu testen ob der Aufbau dicht ist
	\item Mische das 12-\% Trenngel in einem \SI{15}{\ml} Falcon Tube:
		\begin{itemize}
			\item \SI{4}{\ml} Acrylamid-Bisacrylamid-Lösung (30\%)
			\item \SI{2.5}{\ml} Trenngelpuffer
			\item \SI{3.5}{\ml} \ce{\text{dd}H2O}
		\end{itemize}
	\item Mische durch mehrmaliges Invertieren des Röhrchens
	\item Füge \SI{100}{\ul} 10\% APS und \SI{10}{\ul} TEMED hinzu, mische durch
		Invertieren. Polymerisationsmischung unverzüglich bis zu der
		mit Filzstift markierten Höhe zwischen die Glasplatten giessen
	\item Überschichte die Polymerisationslösung sofort mit Isopropanol (\SI{1}{\ml})
	\item Während 20 Minuten polymerisieren lassen
	\item Isopropanol abgiessen und mit ddH2O spülen
	\item Mische das 4.5-\% Sammelgel in einem \SI{15}{ml} Falcon Röhrchen:
		\begin{itemize}
			\item \SI{0.75}{\ml} Acrylamid-Bisacrylamid-Lösung (30\%)
			\item \SI{1.25}{\ml} Sammelgelpufer
			\item \SI{3.0}{\ml} \ce{\text{dd}H2O}
		\end{itemize}
	\item Mische durch mehrmaliges Invertieren des Röhrchens.
	\item Füge \SI{100}{\ul} 10\% APS und \SI{10}{\ul} TEMED hinzu, mische
		durch Invertieren. Überschichte das Sammelgel sofort mit der
		Polymerisationsmischung und setze den Kamm ein
	\item Währen 20 Minuten polymerisieren lassen
\end{itemize}

\section{Vorbereitung Gelelektrophorese}

Anschliessend wurde die Gelelektrophorese durchgeführt. Hierzu wurde der
Gel-Tank aufgebaut und befüllt, und folgende Proben in das Gel gegeben:
\\

\begin{tabu}{ll}
	\toprule
	Lane & Probe \\
	\midrule
	1 & \SI{15}{\ul} unstained marker \\
	2 & \SI{20}{\ul} Input \\
	3 & \SI{30}{\ul} Probe 1 \\
	4 & \SI{30}{\ul} Probe 2 \\ 
	5 & \SI{30}{\ul} Probe 3 \\
	6 & \SI{30}{\ul} Probe 4 \\
	\bottomrule
\end{tabu}
\\

Als unstained marker diente ein Proteingemisch von Proteinen mit bekannter
Grösse, als Input \SI{2}{\ul} des Substrats in \SI{200}{\ul} SDS Buffer,
erhitzt auf \SI{95}{\celsius} für 5 Minuten.

Im Anschluss wurde das Gel für 2.5 Stunden bei \SI{35}{\milli\ampere} laufen
gelassen.

\section{Weiterverarbeitung des Gels}

Im Anschluss wurde das Gel für die Auswertung vorbereitet:
\begin{itemize}
	\item Für 15 Minuten in Fixierlösung inkubiert
	\item Für eine Stunde in Farbstoff Coomassie gegeben
\end{itemize}

Der Rest der Auswertung wurde durch die Assistentin durchgeführt
\cite{skriptv7}:
\begin{itemize}
	\item Wasche kurz mit \ce{\text{dd}H2O}
	\item Inkubiere das Gel für 10 Minuten in \ce{\text{dd}H2O}
	\item Inkubiere das Gel für 15 Minuten in 10\% Glycerol
	\item Lege das Gel umgedreht auf ein Stück Klarsichtfolie und bedecke
		es mit einem Stück 
	\item angefeuchtetem Whatman Filterpapier
	\item Trockne das Gel auf dem Vakuumtrockner bei \SI{60}{\celsius} für 90 Minuten
	\item Entferne die Klarsichtfolie
	\item Markiere den Marker mit kleinen Tropfen aus \SI{1}{\ul}
		\ce{^{35}S-\text{Met}} in \SI{500}{\ul} 50\% Glycerol. Decke
		die Tropfen mit Klebeband ab
	\item Das Gel wird für 12 h einer Phosphoimagerplatte exponiert und am
		nächsten Morgen mit dem Phosphoimager gescannt
\end{itemize}

\chapter{Resultate}

\section{Gefärbtes Gel}

Im Coomassie-gefärbten Gel in Abbildung \ref{fig:gel} wurden Proteine durch den
Farbstoff Coomassie-Blau sichtbar gemacht. Die Konzentration an Proteinen in
einem Punkt korreliert dabei direkt mit der Intensität der blauen Farbe.

Bei allen Proben sind unterhalb der \SI{66}{\kilo\dalton} Linie grosse Blobs zu
sehen. Hierbei wird es sich um BSA handeln, dessen molekulare Masse circa
\SI{60}{\kilo\dalton} beträgt. ADHIII wird aufgrund dessen Molekulargewicht von
\SI{40}{\kilo\dalton} unterhalb der \SI{45}{\kilo\dalton} Linie zu liegen
kommen, was im Autoradiogramm ersichtlich sein wird. Aufgrund der relativ
kleinen Menge an
Proteinen ist es im eingefärbten Gel allerdings nicht dominant sichtbar.

\section{Autoradiogramm}

Im Autoradiogramm in Abbildung \ref{fig:gel_rad} sind radioaktive Nukleide als
schwarze Punkte ersichtlich, wobei die Skala von Weiss nach Schwarz der
Konzentration der radioaktiven Nukleide in jenem Punkt entspricht.

In der Marker-Spalte sind die verschiedenen radioaktiv markierten
Marker-Moleküle mit bekannter Grösse sichtbar.

\subsection{Input-Sample}

Beim Input-Sample befindet sich, wie erwartet knapp unter der
\SI{45}{\kilo\dalton} Linie, das zugegebene, radioaktiv markierte, ADHIII.

\subsection{Probe 1, ohne AVO, ohne Proteinase K}

In der Probe 1 fand, da kein AVO zugegeben wurde, der Proteinimport statt.
Diesen importierten Proteinen wurde in der mitochondrialen Matrix die
N-terminale Sequenz abgeschnitten, die Proteine also kleiner. Da zusätzlich
keine Proteinase K zugegeben wurde, blieben die Proteine, welche nicht
importiert wurden, erhalten. Aufgrund des Grössenunterschiedes verursacht durch
die N-Terminale Sequenz ergeben sich so zwei Banden von Proteinen mit leicht
unterschiedlicher Grösse.

\subsection{Probe 2, mit AVO, ohne Proteinase K}

In der Probe 2 fand aufgrund des zugegebenen AVO kein Proteinimport statt. Da
weiterhin keine Proteinase zugegeben wurde blieben das extrazelluläre ADHIII
erhalten, und befindet sich erwartungsgemäss auf einer Linie mit dem
extrazellulären ADHIII der Probe 1.

\subsection{Probe 3, ohne AVO, mit Proteinase K}

In der Probe 3 wurden Proteine aufgrund nicht vorhandenem AVO importiert, und
nicht importierte Proteine durch die zugegebene Proteinase K abgebaut. Dies
führt zu einer Bande, deren Position bestätigt dass es sich um ADHIII ohne
N-terminale Sequenz handelt. Diese Bande ist allerdings schwacher als die
entsprechende Bande der Probe 1, was daher stammen mag, dass die Proteinase K
entweder gewisse der Mitochondrien abbaute, oder die zugegebene Menge von PMSF
nicht ausreichend war, um alle Proteinase K zu inaktivieren, und somit nach
Aufbrechen der Mitochondrien gewisse der importierten Proteine abgebaut wurden.

\subsection{Probe 4, mit AVO, mit Proteinase K}

In der Probe 4 fand aufgrund des zugegebenen AVO kein Import statt, und
aufgrund der Proteinase K wurden extrazelluläre Proteine abgebaut. Die schwache
Bande auf Höhe des originalen ADHIII mit N-terminaler Sequenz stammt daher,
dass AVO lediglich den zweiten Schritt der Importreaktion - vom
Intermembranraum in das Zellinnere - inhibiert. Der erste Schritt - wie in der
Theorie erwähnt - findet passiv, vermittelt durch die Ladung der N-terminalen
Gruppe und des TOM Komplexes, ohne Verbrauch von ATP, statt. Dies führt dazu,
dass eine kleine Menge des ADHIII in den Intermembranraum importiert wurde, und
dort durch die Membran des Mitochondrium von der Proteinase K geschützt war.

\subsection{Unsaubere Banden}

Allen Banden ist gemein, dass sie nicht gerade, sondern vielmehr gebogen sind.
Die Ursache hierfür ist unklar. Da diese Bögen aber lediglich bei zwei von drei
Gruppen vorhanden sind, wird es vermutlich nicht am Gel liegen. Weiter deuten
die von Probe zu Probe verschiedenen Bögen darauf hin, dass das Problem auch
nicht am Spannungsgerät liegt.

Potentiell wurde bei der Zugabe der Proben in die Taschen des Gels nicht sauber
gearbeitet, und gewisse Taschen verletzt, oder die Proben zu tief eingespritzt.

\begin{figure}[h]
	\centering
	\includegraphics[width=0.9\textwidth]{img/gel.png}
	\caption{Gefärbtes SDS-Gel}
	\label{fig:gel}
\end{figure}

\begin{figure}[h]
	\centering
	\includegraphics[width=0.9\textwidth]{img/gel_rad.png}
	\caption{Autoradiogramm des SDS-Gel}
	\label{fig:gel_rad}
\end{figure}

\bibliographystyle{plain}
\bibliography{references}

\end{document}
