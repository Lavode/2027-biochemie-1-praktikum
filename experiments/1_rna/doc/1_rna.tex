\documentclass[a4paper,english]{scrreprt}

% Uncomment to optimize for double-sided printing.
% \KOMAoptions{twoside}

% Set binding correction manually, if known.
% \KOMAoptions{BCOR=2cm}

% Localization options
\usepackage[german]{babel}
\usepackage[T1]{fontenc}
\usepackage[utf8]{inputenc}

% Enhanced verbatim sections. We're mainly interested in
% \verbatiminput though.
\usepackage{verbatim}

% PDF-compatible landscape mode.
% Makes PDF viewers show the page rotated by 90°.
\usepackage{pdflscape}

% Advanced tables
\usepackage{tabu}
\usepackage{longtable}
\usepackage{dcolumn}
\newcolumntype{d}[1]{D{.}{\cdot}{#1} }

% Fancy tablerules
\usepackage{booktabs}

% Graphics
\usepackage{graphicx}

% Current time
\usepackage[useregional=numeric]{datetime2}

% Float barriers.
% Automatically add a FloatBarrier to each \section
\usepackage[section]{placeins}

% Custom header and footer
\usepackage{fancyhdr}

\usepackage{geometry}
\usepackage{layout}

% Math tools
\usepackage{mathtools}
% Math symbols
\usepackage{amsmath,amsfonts,amssymb}
\usepackage{amsthm}

% SI units
\usepackage{siunitx}
\DeclareSIUnit\Molar{\textsc{m}}
\DeclareSIUnit\rpm{\textsc{rpm}}

% Chemistry
\usepackage{mhchem}

% Subfigures & captions
\usepackage{subcaption}
\usepackage{wrapfig}

\DeclarePairedDelimiter\abs{\lvert}{\rvert}

\pagestyle{plain}
% \fancyhf{}
% \lhead{}
% \lfoot{}
% \rfoot{}
% 
% Source code & highlighting
\usepackage{listings}

% Convenience commands
\newcommand{\mailsubject}{2027 - Lab course biochemistry 1}
\newcommand{\maillink}[1]{\href{mailto:#1?subject=\mailsubject}
                               {#1}}

% Should use this command wherever the print date is mentioned.
\newcommand{\printdate}{\today}

\subject{2027 - Lab course biochemistry 1}
\title{1 - Prozessierung der pre-tRNA mit RNAse P Ribozym}

\author{Michael Senn \maillink{michael.senn@students.unibe.ch} - 16-126-880 - Group 14}

\date{\printdate}

% Needs to be the last command in the preamble, for one reason or
% another. 
\usepackage{hyperref}


\begin{document}
\maketitle

\chapter{Einleitung}

Ziel des Experimentes war die Unterschung der Prozessierung von pre-tRNA durch
RNAse P unter verschiedenen Bedingungen.

Hierzu wurden pre-tRNA und RNAse P in-vivo transkribiert, unter verschiedenen
Bedingungen zusammengegeben, und die Resultierenden Mischungen in einer
Gelelektrophorese analysiert.

\section{Theorie}

\subsection{Katalysierung durch RNA Enzyme}

RNA Enzyme verwenden die gleichen drei Mechanismen um Reaktionen zu
katalysieren die auch Protein-Enzyme verwenden. Dies sind Säure/Base
katalysierte Reaktionen in denen Komponenten der Reaktion (de)protoniert
werden, die elektrostatische Stabilisierung von Zwischenprodukten der Reaktion,
und die Katalyse via optimale Positionierung der an der Reaktion beteiligten
Gruppen.

\subsection{Prozessierung der pre-tRNA durch RNAse}

RNAse P spielt eine wichtige Rolle in der Reifung der tRNAs. Sie katalysiert
durch Hydrolyse die Abspaltung eines 5'-Endes einer Vorläuferform der tRNA. Das
3'-Ende wird im Anschluss, je nach Organsimus, durch weitere RNAsen ebenfalls
abgespalten.

\chapter{Methoden}

\section{In vitro transkription von pre-tRNA \& RNAse P}

In einer in vitro Transkription wurden pre-tRNA beziehungsweise RNAse P
basierend auf EcoRI linearisierten Plasmiden hergestellt. Hierzu wurden
linearisierten Plasmide, Oligonucleotide und eine Polymerase inkubiert, und
anschliessend aufgereinigt.

\section{Elektrophorensen}

Zwecks Kontrolle der Transkriptionsreaktion, sowie Analyse der RNAse P
Aktivität unter verschiedenen Bedingungen, wurden zwei Gelelektrophorsen
durchgeführt.

Zum Nachweis der Produkte der Transkriptionsreaktion diente ein
\SI{1.5}{\percent} Agarose-Gel. Dieses relativ grobporige Gel erlaubte eine
rasche Überprüfung der Transkriptionsresultate, in welchen nur Plasmide,
pre-tRNA, und RNAse P erwartet wurden.

Zur Analyse der RNAse P Aktivität diente ein feinporigeres \SI{8}{\percent}
Polyacrylamid-Ureal Gel. Die denaturierende Eigenschaft von Urea stellte sicher
dass die wandernden Moleküle eine gleichmässige Ladungsverteilung besassen,
sodass sie nur basierend auf ihrer Grösse aufgetrent wurden.

\section{Protokoll}



\chapter{Resultate}

% Messungen, Berechnungen, Resultate

\chapter{Diskussion}

% Evaluiere / bewerte Ziel
% Diskutiere Kontroll-Experimente
% Fragen aus Skript
% Fehlerquellen etc



\bibliographystyle{plainnat}
\bibliography{references}

\end{document}
