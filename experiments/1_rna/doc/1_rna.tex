\documentclass[a4paper,english]{scrreprt}

% Uncomment to optimize for double-sided printing.
% \KOMAoptions{twoside}

% Set binding correction manually, if known.
% \KOMAoptions{BCOR=2cm}

% Localization options
\usepackage[german]{babel}
\usepackage[T1]{fontenc}
\usepackage[utf8]{inputenc}

% Enhanced verbatim sections. We're mainly interested in
% \verbatiminput though.
\usepackage{verbatim}

% PDF-compatible landscape mode.
% Makes PDF viewers show the page rotated by 90°.
\usepackage{pdflscape}

% Advanced tables
\usepackage{tabu}
\usepackage{longtable}
\usepackage{dcolumn}
\newcolumntype{d}[1]{D{.}{\cdot}{#1} }

% Fancy tablerules
\usepackage{booktabs}

% Graphics
\usepackage{graphicx}

% Current time
\usepackage[useregional=numeric]{datetime2}

% Float barriers.
% Automatically add a FloatBarrier to each \section
\usepackage[section]{placeins}

% Custom header and footer
\usepackage{fancyhdr}

\usepackage{geometry}
\usepackage{layout}

% Math tools
\usepackage{mathtools}
% Math symbols
\usepackage{amsmath,amsfonts,amssymb}
\usepackage{amsthm}

% SI units
\usepackage{siunitx}
\DeclareSIUnit\Molar{\textsc{m}}
\DeclareSIUnit\rpm{\textsc{rpm}}

% Chemistry
\usepackage{mhchem}

% Subfigures & captions
\usepackage{subcaption}
\usepackage{wrapfig}

\DeclarePairedDelimiter\abs{\lvert}{\rvert}

\pagestyle{plain}
% \fancyhf{}
% \lhead{}
% \lfoot{}
% \rfoot{}
% 
% Source code & highlighting
\usepackage{listings}

% Convenience commands
\newcommand{\mailsubject}{2027 - Lab course biochemistry 1}
\newcommand{\maillink}[1]{\href{mailto:#1?subject=\mailsubject}
                               {#1}}

% Should use this command wherever the print date is mentioned.
\newcommand{\printdate}{\today}

\subject{2027 - Lab course biochemistry 1}
\title{1 - Prozessierung der pre-tRNA mit RNAse P Ribozym}

\author{Michael Senn \maillink{michael.senn@students.unibe.ch} - 16-126-880 - Group 14}

\date{\printdate}

% Needs to be the last command in the preamble, for one reason or
% another. 
\usepackage{hyperref}


\begin{document}
\maketitle

\chapter{Einleitung}

Ziel des Experimentes war die Untersuchung der Prozessierung von pre-tRNA durch
RNAse P unter verschiedenen Bedingungen.

Hierzu wurden pre-tRNA und RNAse P in-vivo transkribiert, unter verschiedenen
Bedingungen zusammengegeben, und die Resultierenden Mischungen in einer
Gelelektrophorese analysiert.

\section{Theorie}

\subsection{Katalysierung durch RNA Enzyme}

RNA Enzyme verwenden die gleichen drei Mechanismen um Reaktionen zu
katalysieren die auch Protein-Enzyme verwenden. Dies sind Säure/Base
katalysierte Reaktionen in denen Komponenten der Reaktion (de)protoniert
werden, die elektrostatische Stabilisierung von Zwischenprodukten der Reaktion,
und die Katalyse via optimale Positionierung der an der Reaktion beteiligten
Gruppen.

\subsection{Prozessierung der pre-tRNA durch RNAse}

RNAse P spielt eine wichtige Rolle in der Reifung der tRNAs. Sie katalysiert
durch Hydrolyse die Abspaltung eines 5'-Endes einer Vorläuferform der tRNA. Das
3'-Ende wird im Anschluss, je nach Organsimus, durch weitere RNAsen ebenfalls
abgespalten. Der erste, hier untersuchte, Schritt resultiert in einer
Aufteilung der 93bp langen pre-tRNA in ein 75bp langes Produkt, und einen 18bp
langen Abschnitt.

\chapter{Methoden}

\section{In vitro transkription von pre-tRNA \& RNAse P}

In einer in vitro Transkription wurden pre-tRNA beziehungsweise RNAse P
basierend auf EcoRI linearisierten Plasmiden hergestellt. Hierzu wurden
linearisierten Plasmide, Oligonucleotide und eine Polymerase inkubiert, und
anschliessend aufgereinigt.

\section{Elektrophorensen}

Zwecks Kontrolle der Transkriptionsreaktion, sowie Analyse der RNAse P
Aktivität unter verschiedenen Bedingungen, wurden zwei Gelelektrophorsen
durchgeführt.

Zum Nachweis der Produkte der Transkriptionsreaktion diente ein
\SI{1.5}{\percent} Agarose-Gel. Dieses relativ grobporige Gel erlaubte eine
rasche Überprüfung der Transkriptionsresultate, in welchen nur Plasmide,
pre-tRNA, und RNAse P erwartet wurden.

Zur Analyse der RNAse P Aktivität diente ein feinporigeres \SI{8}{\percent}
Polyacrylamid-Ureal Gel. Die denaturierende Eigenschaft von Urea stellte sicher
dass die wandernden Moleküle eine gleichmässige Ladungsverteilung besassen,
sodass sie nur basierend auf ihrer Grösse aufgetrent wurden. Zwecks Einfärbung
der Fragmente diente Ethidiumdibromid. Anzumerken ist, dass Entidiumdibromid an
das abgeschnittene 5'-Ende zu wenig gut binden kann als dass es im Gel sichtbar
wäre.

\section{Protokoll\cite{skriptv1}}

\subsection{In vitro Transkription von pre-tRNA}

Herstellung einer Transkriptions-Mischung:
\\

\begin{tabu}{ll}
	\toprule
	Stoff & Menge \\
	\midrule
	EcoRI linearisiertes tRNA Plasmid \SI{573}{\ng \per \ul} & \SI{1.745}{\ul} \\
	rNTP Mix (ATP, GTP, UTP, CTP), je \SI{25}{\milli\Molar} & \SI{15}{\ul} \\
	10x TC-Buffer & \SI{10}{\ul} \\
	\ce{H2O} & \SI{70.755}{\ul} \\
	RNasin Plus RNAse Inhibitor 5U & \SI{1}{\ul} \\
	T7-RNA Polymerase & \SI{1.5}{\ul} \\
	\bottomrule
\end{tabu}
\\

Anschliessend Inkubation für \SI{3.5}{\hour} bei \SI{37}{\celsius}.

\subsection{In vitro Transkription von RNAse P}

Analog vorherigem Absatz durch zweite Gruppe hergestellt.

\subsection{Herstellung \SI{1.5}{\percent} Agarose-Gel}

\SI{1.5}{\percent} Agarose-Lösung in 1x TAE mit \SI{0.4}{\ug \per \ml}
\ce{EtBr} wurde in der Mikrowelle geschmolzen, und in die vorbereitete
Gelelektrophorse-Kammer gegossen.

Nach Polymerisierung des Gels wurde es zwecks Haltbarkeit mit 1x TAE
übergossen.

\subsection{Kontrolle der Transkription}

\SI{5}{\ul} des Transkriptionsresultates wurde mit \SI{5}{\ul} 2x RNA loading
dye gemischt und bei \SI{95}{\celsius} für \SI{2}{\min} inkubiert.

Die kompletten \SI{10}{\ul} wurden zusammen mit 5S rRNA als Grössenmarker auf
das Agarose-Gel geladen, und bei \SI{50}{\volt} für \SI{45}{\min} laufen
gelassen. Das resultierende Gel wurde unter UV Licht betrachtet.

Zum Rest der Transkriptionsreaktion wurde \SI{5}{\ul} \SI{1}{U \per \ul} DNAse
I gegeben und bei \SI{37}{\celsius} für ca \SI{45}{\min} inkubiert. Geplant war
eine Inkubationsdauer von \SI{15}{\min}, der Zeitpunkt der Entnahme wurde aber
verpasst.

\subsection{RNAse P Prozessierungsreaktionen}

\subsubsection{Inhibition mit Neomycin B}

Für die Reaktionsserien wurden zwei Mastermixe hergestellt. Mastermix A, mit
von einer Vorgängergruppe hergestellte tRNA:
\\

\begin{tabu}{ll}
	\toprule
	\multicolumn{2}{c}{Mastermix A} \\
	\midrule
	pre-tRNA \SI{10}{\pico\Molar\per\ul} & \SI{4.5}{\ul} \\
	5x RNAse P Buffer & \SI{9}{\ul} \\
	\ce{MgCl2} \SI{300}{\milli\Molar} & \SI{4.5}{\ul} \\
	\ce{H2O} & \SI{27}{\ul} \\
	\bottomrule
\end{tabu}
\\

Mastermix A wurde auf vier Proben verteilt, und diese bei \SI{55}{\celsius} für
\SI{5}{\minute} inkubiert.

Im Anschluss wurde eine erste Version des Mastermix B hergestellt:
\\

\begin{tabu}{ll}
	\toprule
	\multicolumn{2}{c}{Mastermix $B_1$} \\
	\midrule
	5x RNAse P Buffer & \SI{9}{\ul} \\
	\ce{MgCl2} \SI{300}{\milli\Molar} & \SI{4.5}{\ul} \\
	\bottomrule
\end{tabu}
\\

Aus dem Mastermix $B_1$ wurden \SI{3}{\ul} zu \SI{7}{\ul} \ce{H2O} gegeben, um
eine Negativkontrolle (Probe "Neg") zu erhalten. Danach wurde $B_2$, die zweite
Version des Mastermix B, durch Zugabe von \SI{7.7}{\ul}
\SI{4.5}{\pico\Molar\per\ul} RNAse P erzeugt.

Hiervon wurden je \SI{5.2}{\ul} auf drei Proben verteilt, und mit \SI{3}{\ul}
\SI{6}{\milli\Molar} Ampicilin \& \SI{1.8}{\ul} \ce{H2O} (Probe "Amp"),
\SI{4.8}{\ul} \ce{H2O} (Probe "Neo$-$"), beziehungsweise \SI{3}{\ul}
\SI{6}{\milli\Molar} Neomycin \& \SI{1.8}{\ul} \ce{H2O} (Probe "Neo$+$")
versetzt.  Auch diese Proben wurdeb bei \SI{55}{\celsius} für \SI{5}{\minute}
inkuibiert.

Im Anschluss wurden jeweils eine der Proben mit Mastermix $A$ und eine der mit
Mastermix $B$ gemischt und bei \SI{37}{\celsius} für \SI{30}{\min} inkubiert.
Die Reaktion wurde durch \SI{20}{\ul} 2x RNA loading dye gestoppt und die
Proben bei \SI{95}{\celsius} für \SI{2}{\min} inkubiert. Schlussendlich wurden
sie auf Eis gestellt.

Zusammenfassend wurden also folgende vier Proben vorbereitet, gemischt, und
inkubiert:
\\

\begin{tabu}{lrrrr}
	\toprule
	                & Neg          & Neo$-$         & Neo$+$        & Amp \\
	\midrule
	pre-tRNA        & \SI{1}{\ul}  & \SI{1}{\ul}    & \SI{1}{\ul}   & \SI{1}{\ul} \\
	5x RNAse Puffer & \SI{4}{\ul}  & \SI{4}{\ul}    & \SI{4}{\ul}   & \SI{4}{\ul} \\
	\ce{MgCl2}      & \SI{2}{\ul}  & \SI{2}{\ul}    & \SI{2}{\ul}   & \SI{2}{\ul} \\
	RNAse P         &              & \SI{2.2}{\ul}  & \SI{2.2}{\ul} & \SI{2.2}{\ul} \\
	Ampicilin       &              &                &               & \SI{3}{\ul} \\
	Neomycin        &              &                & \SI{3}{\ul}   &             \\
	\ce{H2O}        & \SI{13}{\ul} & \SI{10.8}{\ul} & \SI{7.8}{\ul} & \SI{7.8}{\ul} \\
	\bottomrule
\end{tabu}

\subsubsection{Einfluss der Inkubationstemperatur \& Magnesiumkonzentration}

Durch die zweite Gruppe respektive die Assistentin wurden weitere zwei
Messreihen durchgeführt.

\subsection{Denaturierende PAA Gelelektrophorese}

Zwecks Analyse der Resultate der RNA-Prozessierungsreaktion wurde eine
denaturierende Polyacrylamid-Gelelektrophorese durchgeführt.

\subsubsection{Herstellung \SI{8}{\percent} Polyacrylamid-Urea-Gel}

Die Glasplatten der Laufkammer wurden mit Seife und Ethanol gereinigt und die
Kammer zusammengesetzt. \SI{70}{\ml} einer \SI{8}{\percent}
Polyacrylamid-Urea-Lösung wurden mit \SI{600}{\ul} \SI{10}{\percent} APS, und
\SI{30}{\ul} TEMED gemischt. Im Anschluss wurde die Lösung in die Kammer
gegossen, und polymerisiert.

\subsubsection{Gelelektrophorese}

Das Geld wurde gemäss folgendem Ladeschema bestückt, und für \SI{2}{\hour}
\SI{40}{\min} bei \SI{300}{\V} laufen gelassen. Im Anschluss wurde die RNA mit
Ethidiumbromid für \SI{5}{\min} markiert, und das Gel unter UV Licht
fotografiert.
\\

\begin{tabu}{ll}
	\toprule
	Kammer & Probe \\
	\midrule
	1 & Size Marker \\
	\midrule
	\multicolumn{2}{c}{Inkubationszeit} \\
	\cmidrule(lr){1-2}
	2 & Neg \\
	3 & \SI{1}{\min} \\
	4 & \SI{15}{\min} \\
	5 & \SI{30}{\min} \\
	\midrule
	\multicolumn{2}{c}{\ce{Mg}-Konzentration} \\
	\cmidrule(lr){1-2}
	6 & Neg \\
	7 & \SI{0.5}{\milli\Molar} \\
	8 & \SI{1}{\milli\Molar} \\
	9 & \SI{30}{\milli\Molar} \\
	\midrule
	\multicolumn{2}{c}{Inhibierung mit Antibiotika} \\
	\cmidrule(lr){1-2}
	10 & Neg \\
	11 & Neo$-$ \\
	12 & Neo$+$ \\
	13 & Amp \\
	\bottomrule
\end{tabu}

\chapter{Resultate}

% Messungen, Berechnungen, Resultate

\section{Kontrolle der Transkription}

Abbildung \ref{fig:transkription_kontrolle} zeigt die Resultate der
Transkriptionskontrolle unter UV Licht. 

In der Spalte der RNAse P ist eine Bande im Bereich des 300bp Markers
ersichtlich, in der Spalte der pre-tRNA eine Bande kurz unterhalb des 100bp
Markers sowie eine weitere feine Bande kurz unterhalb des 1000bp Markers.

Die erwartete Länge der pre-tRNA beträgt 93bp, was sich gut mit der aus dem Gel
ablesbaren Länge des Produktes deckt. Die erwartete Länge der RNAse P beträgt
382bp, was leicht grösser als die scheinbare Länge des Produktes ist. Die feine
Bande bei 1000bp entspricht dem zugegebenen Plasmid.

Die Transkription der pre-tRNA scheint damit erfolgreich gewesen zu sein, bei
der RNAse P besteht eine kleine Unsicherheit.

\begin{figure}[h]
	\centering
	\includegraphics[width=0.8\textwidth]{img/gel_transkription.png}
	\caption{Kontrolle der tRNA \& RNAse P Transkription, Agarose-Gel unter UV Licht, 2x RNA loading dye-gefärbt}
	\label{fig:transkription_kontrolle}
\end{figure}

\section{RNAse P Prozessierungsreaktionen}

Abbildung \ref{fig:rnase_prozessierungsreaktionen} zeigt die Resultate der
Analyse der Prozessierungsreaktion unter verschiedenen Bedingungen. Das Gel
wurde entsprechend dem Protokoll mit Ethidiumdibromid gefärbt, und unter UV
Licht fotografiert.

Die schwachen weissen Banden am unteren Rand des Gels, sowie in gewissen
Spalten, entsprechen der loading dye mit bekannten Grössen von 12bp respektive
75bp. Basierend darauf, sowie auf dem verwendeten size marker, konnten die
Grössen der restlichen Banden abgeschätzt werden.

\begin{figure}[h]
	\centering
	\includegraphics[width=\textwidth]{img/gel_rnase.png}
	\caption{Analyse der RNAse P Prozessierungsreaktionen, Denaturierendes PAA-Geal unter UV Licht, markiert mit Ethidiumdibromid}
	\label{fig:rnase_prozessierungsreaktionen}
\end{figure}


\subsection{Negativkontrollen}

In allen Negativkontrollen ist bei 387bp keine Bande zu sehen, was bestätigt
dass diese keine RNAse P enthielten. Wie erwartet fand dadurch keine
Prozessierung der pre-tRNA statt, sodass bei 93pb die Bande der pre-tRNA
sichtbar ist, aber keine weitere die der prozessierten pre-tRNA entsprechen
würde.

\subsection{Inhibierung mit Antibiotika}

Ohne Zugabe von Neomycin fand wie erwartet eine Prozessierung der pre-tRNA
statt. Jene wurde komplett durch die bei ca 300bp sichtbare RNAse P umgesetzt,
und befindet sich auf Höhe der 75bp loading dye.

Unter Zugabe von Neomycin ist eine Bande bei 93bp, aber keine bei ca 75bp
ersichtlich. Dies bestätigt dass Neomycin die Prozessierung der pre-tRNA
inhibiert. Die tiefere Konzentration der RNAse P in dieser Probe lässt sich
aufgrund Verwendung eines Mastermix nicht auf einen Pipettierfehler
zurückführen, sondern wird vermutlich das Resultat einer im kommerziell
erworbenen Neomycin vorhandenen Protease sein, welche die RNAse P über Zeit
abbaute.

Unter Zugabe von Ampicilin ist eine Doppelbande - vermutlich durch einen
Laufeffekt - auf Höhe der 75bp loading dye ersichtlich. Ampicilin inhibiert die
Aktivität der RNAse P also nicht, die Prozessierung der pre-tRNA fand somit
statt.

\subsection{Effekt der Inkubationszeit \& \ce{Mg}-Konzentration}

Eine Verlängerung der Inkubationszeit von \SI{1}{\min} auf \SI{15}{\min} führte
zu einer minimale stärkeren Konzentration der Endprodukte, eine weitere
Verlängerung auf \SI{30}{\min} hatte aber einen vernachlässigbaren Effekt.

Der Einfluss der \ce{Mg}-Konzentration ist klar ersichtlich, mit nur minimaler
Umsetzung bei \SI{0.5}{\milli\Molar}, einer Umsetzung von circa der Hälfte bei
\SI{1}{\milli\Molar}, und einer fast kompletten Umsetzung bei
\SI{30}{\milli\Molar}.

\chapter{Diskussion}

\section{Run-off Transkription}

Bei der run-off Transkription bindet die Polymerase am Promoter eines
linearisierten DNA Fragmentes, und transkribiert so lange bis sie vom Ende des
Fragmentes abfällt. Dies ermöglicht beispielsweise Transkription von Fragmenten
bestimmter Länge, ohne dass am Ende dieser Fragmente ein Terminator vorkommen
muss, oder auch die Transkription eines Fragmentes mit rho-abhängigem
Terminator ohne dass Zugabe von Rho notwendig ist.

\section{Einfluss der untersuchten Faktoren}

\subsection{Inkubationszeit}

Eine Verlängerung der Inkubationszeit führte zu minimal stärkerem Umsatz, wobei
der Effekt ab \SI{15}{\min} vernachlässigbar ist.

\subsection{\ce{Mg}-Konzentration}

Eine Erhöhung der \ce{Mg}-Konzentration, welche der RNAse P als Kofaktor dient,
führte zu stärkerem Umsatz. Ab \SI{30}{\milli\Molar} konnte ein fast kompletter
Umsatz beobachtet werden.

\subsection{Inhibierung mit Antibiotika}

Zugabe von \SI{3}{\milli\Molar} Neomycin B führte zu einer kompletten
Inhibierung der Prozessierung. Zugabe von Ampicilin hatte keinen solchen
Effekt.

\section{Effekt von Neomycin B auf menschliche RNAse P}

Gesetzt den Fall dass Neomycin die Zelle betreten kann, ist davon auszugehen
dass Neomycin B die menschliche nukleare RNAse P ebenfalls durch Verdrängung
der Magensium-Ionen inhibiert.\cite{inhibition_eukaryotic_rnasep}. Die
mitochondriale RNAse P, welche nur noch aus Proteinkomponenten besteht, hat
potentiell einen anderen Wirkungsmechanismums und wäre damit nicht betroffen.

\section{Inhibierungsmechanismums von Neomycin B \& Ampicilin}

Neomycin B agiert als nicht-kompetitivier Inhibitor, und besetzt zwei der
\ce{Mg^{2+}} Bindungsstellen der RNAse P, wodurch deren Aktivität gehindert
wird.\cite{skriptv1}

Im Vergleich dazu inhibiert Ampicilin die Synthese der Zellwand in
Gram-positiven und gewissen Gram-negativen Bakterien, was ultimativ zur Lysis
führt.\cite{website:pubchem_ampicilin}.

Damit ist auch klar, dass Ampicilin keinen direkten Effekt auf RNAse P hat, was
im Experiment bestätigt wurde.

\section{Katalysierungsreaktion von RNAse P mit pre-tRNA Substrat}


RNAse P erkennt und bindet an das 3' Ende der pre-tRNA mittels einer
spezifischen Sequenz. Ein Magnesium-Wasserstoff Komplex dient dann als
Nukleophil für die Hydrolyse einer Phosphordiesterbindung am 5' Ende. Das
Abgangsprodukt wird durch einen weiteren Magensium-Wasserstoff Komplex
stabilisiert.

\bibliographystyle{vancouver}
\bibliography{references}

\end{document}
