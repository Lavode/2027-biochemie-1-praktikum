\documentclass[a4paper,german]{scrreprt}

% Uncomment to optimize for double-sided printing.
% \KOMAoptions{twoside}

% Set binding correction manually, if known.
% \KOMAoptions{BCOR=2cm}

% Localization options
\usepackage[german]{babel}
\usepackage[T1]{fontenc}
\usepackage[utf8]{inputenc}

% Enhanced verbatim sections. We're mainly interested in
% \verbatiminput though.
\usepackage{verbatim}

% PDF-compatible landscape mode.
% Makes PDF viewers show the page rotated by 90°.
\usepackage{pdflscape}

% Advanced tables
\usepackage{tabu}
\usepackage{longtable}
\usepackage{dcolumn}
\newcolumntype{d}[1]{D{.}{\cdot}{#1} }

% Fancy tablerules
\usepackage{booktabs}

% Graphics
\usepackage{graphicx}

% Current time
\usepackage[useregional=numeric]{datetime2}

% Float barriers.
% Automatically add a FloatBarrier to each \section
\usepackage[section]{placeins}

% Custom header and footer
\usepackage{fancyhdr}

\usepackage{geometry}
\usepackage{layout}

% Math tools
\usepackage{mathtools}
% Math symbols
\usepackage{amsmath,amsfonts,amssymb}
\usepackage{amsthm}

% SI units
\usepackage{siunitx}
\DeclareSIUnit\Molar{\textsc{m}}
\DeclareSIUnit\rpm{\textsc{rpm}}

% Chemistry
\usepackage{mhchem}

% Subfigures & captions
\usepackage{subcaption}

\DeclarePairedDelimiter\abs{\lvert}{\rvert}

\pagestyle{plain}
% \fancyhf{}
% \lhead{}
% \lfoot{}
% \rfoot{}
% 
% Source code & highlighting
\usepackage{listings}

% Convenience commands
\newcommand{\mailsubject}{2027 - Praktikum Biochemie 1}
\newcommand{\maillink}[1]{\href{mailto:#1?subject=\mailsubject}
                               {#1}}

% Should use this command wherever the print date is mentioned.
\newcommand{\printdate}{\today}

\subject{2027 - Praktikum Biochemie 1}
\title{3 - Real-time quantitative PCR}

\author{Michael Senn \maillink{michael.senn@students.unibe.ch} - 16-126-880 - Group 14}

\date{\printdate}

% Needs to be the last command in the preamble, for one reason or
% another. 
\usepackage{hyperref}


\begin{document}
\maketitle

\chapter{Introduction}

\chapter{Protocol}

\section{Sample preparation}

A cell suspension was prepared ahead of time, containing $2.5 \cdot 10^5$ cells
per \SI{180}{\ul}. A dilution of neuraminidase in glycobuffer was created, with
a concentration of \SI{0.5}{U \per \ul}.

Then, four samples were prepared according to the following table.
\\

\begin{tabu}{lllll}
	\toprule
	Sample & N-1 & N-2 & N+1 & N+2 \\
	\midrule
	Cell suspension & \SI{180}{\ul} & \SI{180}{\ul} & \SI{180}{\ul} & \SI{180}{\ul} \\
	Neuraminidase dilution & \SI{0}{\ul} & \SI{0}{\ul} & \SI{20}{\ul} & \SI{20}{\ul} \\
	Glycobuffer & \SI{20}{\ul} & \SI{20}{\ul} & \SI{0}{\ul} & \SI{0}{\ul} \\
	\bottomrule
\end{tabu}
\\

The four samples were mixed on a vortexer, and incubated for \SI{30}{\minute}
at \SI{37}{\celsius}, to facilitate activity of the neuraminidase.

\section{Virus internalisation}

\SI{10}{\ul} of a MVM-infected supernatant, with a concentration of $10^7$
virions per \si{\ul}, were added to all samples. The samples were then mixed on
a vortexer, and incubated for \SI{1}{\hour} at \SI{37}{\celsius}.

\section{Washing the samples}

In order to remove excess virus, the samples were washed thrice. During each
cycle they were first centrifuged at \SI{3000}{g} for \SI{3}{\min}, the
supernatant then discarded and the pellets resuspended in \SI{900}{\ul} ice-cold
PBS. After the final washing cycle, the supernatant was removed.

\section{Lysis with Chelex}

Through rapid pipetting, a provided chelex suspension of \SI{100}{\mg \per \ml}
was homogenized, and \SI{500}{\ul} transferred to a separate tube. This
solution was homogenized once more, and \SI{100}{\ul} were added to each
sample. The samples were vortexed, and incubated for \SI{10}{\min} at
\SI{95}{\celsius}, while shaking at \SI{400}{\rpm}.

Afterwards the samples were cooled down to prevent vapor build-up, vortexed
once more, and centrifuged for \SI{1}{\min} at \SI{15000}{\rpm}.

\SI{25}{\ul} of the supernatant, containing the DNA, was added to a separate
tube.

\section{Preparation of qPRC master-mix}

Two types of qPRC master-mix were prepared. The first one contained iTaq-mix,
forward primer and reverse primer, the second one was identical except it did
not contain reverse primer.

Composition of the first primer:
\\

\begin{tabu}{lll}
	\toprule
	Component & Fractions & Volume \\
	\midrule
	iTaq mix & 16 & \SI{112}{\ul} \\
	Forward primer MVM-F & 1 & \SI{7}{\ul} \\
	Reverse primer MVM-R-489 & 1 & \SI{7}{\ul} \\
	\bottomrule
\end{tabu}
\\

Composition of the second primer:
\\

\begin{tabu}{lll}
	\toprule
	Component & Fractions & Volume \\
	\midrule
	iTaq mix & 16 & \SI{112}{\ul} \\
	Forward primer MVM-F & 1 & \SI{7}{\ul} \\
	\bottomrule
\end{tabu}

\section{Preparation of PCR samples}

\SI{18}{\ul} of the first master-mix were added to five wells. Afterwards the
four samples, plus pure \ce{H2O} as a negative control, were added to these
wells.

\SI{17}{\ul} each of the second master-mix were added to six additional wells.
\SI{1}{\ul} of different reverse primers were added, with MVM-R-72 being added
to wells 6 and 7, MVM-R-432 to wells 8 and 9, and MVM-R-1037 to wells 10 and
11. Afterwards \SI{2}{\ul} of the samples without neuraminidase were added to
those wells, with sample 1 being added to wells 6, 8, 10, and sample 2 being
added to wells 7, 9 and 11.

In summary, the following wells were prepared:
\\

\begin{tabu}{lccccccccccc}
	\toprule
	Well & 1 & 2 & 3 & 4 & 5 & 6 & 7 & 8 & 9 & 10 & 11 \\
	\midrule 
	iTaq Mix & \SI{16}{\ul} & \SI{16}{\ul} & \SI{16}{\ul} & \SI{16}{\ul} & \SI{16}{\ul} & \SI{16}{\ul} & \SI{16}{\ul} & \SI{16}{\ul} & \SI{16}{\ul} & \SI{16}{\ul} & \SI{16}{\ul} \\
	MVM-F & \SI{1}{\ul} & \SI{1}{\ul} & \SI{1}{\ul} & \SI{1}{\ul} & \SI{1}{\ul} & \SI{1}{\ul} & \SI{1}{\ul} & \SI{1}{\ul} & \SI{1}{\ul} & \SI{1}{\ul} & \SI{1}{\ul} \\
	MVM-R-489 & \SI{1}{\ul} & \SI{1}{\ul} & \SI{1}{\ul} & \SI{1}{\ul} & \SI{1}{\ul} &  &  &  &  &  &  \\
	MVM-R-72 & & & & & & \SI{1}{\ul} & \SI{1}{\ul} & & & & \\
	MVM-R-432 & & & & & & & & \SI{1}{\ul} & \SI{1}{\ul} & & \\
	MVM-R-1037 & & & & & & & & & & \SI{1}{\ul} & \SI{1}{\ul} \\
	N$+$ 1 & \SI{2}{\ul} & & & & & & & & & & \\
	N$+$ 2 & & \SI{2}{\ul} & & & & & & & & & \\
	N$-$ 1 & & & \SI{2}{\ul} & & & \SI{2}{\ul} & & \SI{2}{\ul} & & \SI{2}{\ul} & \\
	N$-$ 2 & & & & \SI{2}{\ul} & & & \SI{2}{\ul} & & \SI{2}{\ul} & & \SI{2}{\ul} \\
	\ce{H2O} & & & & & \SI{2}{\ul} & & & & & & \\
	\bottomrule
\end{tabu}

\chapter{Results \& discussion}

\bibliographystyle{plain}
\bibliography{references}

\end{document}
