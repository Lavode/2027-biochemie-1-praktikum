\documentclass[a4paper,english]{scrreprt}

% Uncomment to optimize for double-sided printing.
% \KOMAoptions{twoside}

% Set binding correction manually, if known.
% \KOMAoptions{BCOR=2cm}

% Localization options
\usepackage[german]{babel}
\usepackage[T1]{fontenc}
\usepackage[utf8]{inputenc}

% Enhanced verbatim sections. We're mainly interested in
% \verbatiminput though.
\usepackage{verbatim}

% PDF-compatible landscape mode.
% Makes PDF viewers show the page rotated by 90°.
\usepackage{pdflscape}

% Advanced tables
\usepackage{tabu}
\usepackage{longtable}
\usepackage{dcolumn}
\newcolumntype{d}[1]{D{.}{\cdot}{#1} }

% Fancy tablerules
\usepackage{booktabs}

% Graphics
\usepackage{graphicx}

% Current time
\usepackage[useregional=numeric]{datetime2}

% Float barriers.
% Automatically add a FloatBarrier to each \section
\usepackage[section]{placeins}

% Custom header and footer
\usepackage{fancyhdr}

\usepackage{geometry}
\usepackage{layout}

% Math tools
\usepackage{mathtools}
% Math symbols
\usepackage{amsmath,amsfonts,amssymb}
\usepackage{amsthm}

% SI units
\usepackage{siunitx}
\DeclareSIUnit\Molar{\textsc{m}}
\DeclareSIUnit\mole{mol}
\DeclareSIUnit\rpm{\textsc{rpm}}

% Chemistry
\usepackage{mhchem}

% Subfigures & captions
\usepackage{subcaption}
\usepackage{wrapfig}

\DeclarePairedDelimiter\abs{\lvert}{\rvert}

\pagestyle{plain}
% \fancyhf{}
% \lhead{}
% \lfoot{}
% \rfoot{}
% 
% Source code & highlighting
\usepackage{listings}

% Convenience commands
\newcommand{\mailsubject}{2027 - Lab course biochemistry 1}
\newcommand{\maillink}[1]{\href{mailto:#1?subject=\mailsubject}
                               {#1}}

% Should use this command wherever the print date is mentioned.
\newcommand{\printdate}{\today}

\subject{2027 - Lab course biochemistry 1}
\title{2 - Rechtsmedizin: Entführungsfall mit Lösegelderpressung}

\author{Michael Senn \maillink{michael.senn@students.unibe.ch} - 16-126-880 - Gruppe 14}

\date{\printdate}

% Needs to be the last command in the preamble, for one reason or
% another. 
\usepackage{hyperref}


\begin{document}
\maketitle

\chapter{Einführung}

Ziel des Experimentes war es, die Spezies einer Fleischprobe zu bestimmen.
Hierzu wurde die DNA aufgereinigt, eine Sequenz der mitochondrialen DNA mittels
PCR amplifiziert und mit Restriktionsenzymen verdaut. Die resultierenden
Fragmente wurden in einer Gelelektrophorese aufgetrennt, und die Bandenmuster
mit den Mustern bekannter Spezies verglichen.

\chapter{Durchführung\cite{skriptv2}}

\section{Verwendete Chemikalien}

\begin{itemize}
	\item Tris-\ce{HCl} \SI{10}{\milli\Molar}
	\item TNE Extraktionspuffer
		\begin{itemize}
			\item \SI{10}{\milli\Molar} Tris-\ce{HCl}
			\item \SI{150}{\milli\Molar} \ce{NaCl}
			\item \SI{2}{\milli\Molar} EDTA
			\item \SI{1}{\percent} SDS
		\end{itemize}
	\item Guanidin-Hydrochlorid \SI{5}{\Molar}
	\item Proteinase K \SI{20}{\mg \per \ml}
	\item Isopropanol \SI{80}{\percent}
	\item Ethanol \SI{70}{\percent}
	\item DNA Grössenmarker, 100bp Leiter
\end{itemize}

\section{DNA Extraktion}

Zwecks Extraktion der DNA wurde die erhaltene Fleischprobe, sowie eine
Extraktionskontrolle bestehend aus Wasser, wie folgt verarbeitet.

\begin{itemize}
	\item Zur Kontrolle des sauberen Arbeitens bei der DNA-Extraktion wird
		ein weiteres \SI{2}{\ml} Röhrchen mit Wasser (\SI{300}{\ul})
		anstelle von Fleisch aufgearbeitet (Extraktionskontrolle).
	\item \SI{820}{\ul} TNE-Extraktionspuffer, \SI{80}{\ul} Proteinase K
		(\SI{20}{\mg \per \ml}) und \SI{100}{\ul} \SI{5}{\Molar}
		Guanidin- Hydrochlorid-Lösung dazupipettieren und gut mischen
		(vortexen).
	\item \SIrange{1.5}{2}{\hour} im Heizblock bei \SI{60}{\celsius}
		inkubieren.
	\item \SI{2}{\min} abkühlen lassen, dann \SI{10}{\min} bei 16‘000 g
		zentrifugieren
	\item Die \SI{10}{\milli\Molar} Tris-\ce{HCl} (pH 8.0) Lösung auf
		\SIrange{60}{65}{\celsius} erhitzen (Elutionspuffer, siehe
		unten).
	\item In zwei frische Eppis je \SI{500}{\ul} gut resuspendierte
		Glasmilch vorlegen.
	\item \SI{750}{\ul} des Überstandes des Gewebextrakts (Achtung: Fett
		und Fleischkrümel vermeiden!) zu der Glasmilch geben und durch
		mehrmaliges sanftes Kippen des Röhrchens gut mischen (DNA
		bindet an Glasmilch). Beschriften Sie Ihre Proben gut! Sie
		können sich bei Ihrer Analyse keine Verwechslungen leisten.
	\item Beschriftete Säule (mechanischer Filter) auf \SI{2}{\ml}-Spritze
		stecken (Kolben beschriften, herausnehmen und in Verpackung
		zurücklegen).
	\item Glasmilchmischung in die Spritze pipettieren oder giessen.
	\item Mischung mittels Kolben langsam durch die Säule stossen, vor dem
		Herausziehen des Kolbens, Spritze von der Säule entfernen (wg.
		Unterdruck).
	\item Säule mit \SI{2}{\ml} \SI{80}{\percent} Isopropanol waschen:
		\SI{80}{\percent} Isopropanol in Spritze geben und mit Kolben
		durch Säule stossen.
	\item Säule mit \SI{2}{\ml} \SI{70}{\percent} Ethanol waschen wie
		vorher.
	\item Spritzen entfernen und Säulen auf Eppendorfröhrchen stecken.
		Säulen \SI{3}{\min} bei 12'000 g zentrifugieren (Ethanol
		entfernen).
	\item Säulen auf ein neues steriles \SI{1.5}{\ml} Eppendorfröhrchen
		stecken und \SIrange{5}{10}{\min} bei Raumtemperatur trocknen
		lassen (letztes Ethanol verdunsten lassen).
	\item \SI{50}{\ul} heisses \SI{10}{\milli\Molar} Tris-\ce{HCl} pH 8.0
		zugeben zum Eluieren.
	\item \SI{1}{\min} inkubieren und Säulen kurz (\SI{30}{\s}) bei 12'000
		g zentrifugieren.
	\item Eppendorfröhrchen mit mtDNA-Eluat verschliessen und bei
		\SI{4}{\celsius} (in Eis) für weitere Analysen aufbewahren.
	\item Ausbeute und Reinheit spektroskopisch am NanoDrop ermitteln,
		Werte notieren, mtDNAs im Eis aufbewahren (gut beschriftet).
\end{itemize}

\subsection{Reinheitsmessung am Nanodrop}

Im Nanodrop wurde die Absorption der mtDNA sowie der Extraktionskontrolle bei
UV Licht gemessen. Von besonderem Interesse waren dabei die Absorption bei
\SI{260}{\nm} da dies die Wellenlänge ist bei der doppelsträngige DNA Helices
absorbieren, sowie \SI{280}{\nm}. Aus den beiden Messungen und deren Verhältnis
wurden die Reinheit sowie die Konzentration der mtDNA ermittelt.
\\

\begin{tabu}{lll}
	\toprule
	Probe & mtDNA & Extraktionskontrolle \\
	\midrule
	Konzentration & \SI{208.9}{\ng \per \ul} & \SI{6.8}{\ng \per \ul} \\
	Verhältnis Absorption \SI{260}{\nm} : \SI{280}{\nm} & 1.89 & 2.24 \\
	\bottomrule
\end{tabu}
\\

Aus der tiefen DNA Konzentration der Extraktionskontrolle lässt sich schliessen
dass sauber gearbeitet wurde, während das 260 : 280 Verhältnis der mtDNA Probe
über 1.8 andeutet dass die aufgereinigte DNA nur mit wenig Proteinen
verunreinigt ist.

\section{Polymerase Kettenreaktion}

Mittels PCR wurde ein Segment der aufgereinigten mitochondriellen DNA
vervielfacht. Als Positivkontrolle diente humane mitochondrielle DNA.

\subsection{Primersequenzen}

Als Primer wurden cytbL und cytbH gewählt. Diese binden selektiv an
konservierte Regionen des Cytochrom B Gens, und flankieren einen variablen
Bereich, in der mitochondrialen DNA. Die Primer haben folgende Sequenz:

\begin{description}
	\item[cytb1 (cytbL)] 5`- CCA TCC AAC ATC TCA GCA TGA TGA AA -3`
	\item[cytb2 (cytbH)] 5`- GCC CCT CAG AAT GAT ATT TGT CCT CA -3`
\end{description}

\subsection{PCR Mastermix}

Zwecks Verbesserung der Pipettiergenauigkeit wurde ein Mastermix für die PCR
vorbereitet.
\\

\begin{tabu}{llll}
	\toprule
	Stocklösung & Volumen 1x & Volumen 4.5x & Konzentration je Sample \\
	\midrule
	Templat-DNA                        & \SI{1.5}{\ul}  & -               & - \\
	5x PCR Buffer für Taq              & \SI{12}{\ul}   & \SI{54}{\ul}    & 1x \\
	\ce{H2O}                           & \SI{35.4}{\ul} & \SI{159.3}{\ul} & - \\
	\ce{MgCl2} \SI{25}{\milli\Molar}   & \SI{6}{\ul}    & \SI{27}{\ul}    & \SI{2.5}{\milli\Molar} \\
	Primer cytbL \SI{10}{\micro\Molar} & \SI{2}{\ul}    & \SI{9}{\ul}     & \SI{20}{\pico\mole} \\
	Primer cytbH \SI{10}{\micro\Molar} & \SI{2}{\ul}    & \SI{9}{\ul}     & \SI{20}{\pico\mole} \\
	dNTP (je \SI{10}{\milli\Molar})    & \SI{0.6}{\ul}  & \SI{2.7}{\ul}   & \SI{100}{\micro\Molar} \\
	Taq Polymerase \SI{5}{U \per \ul}  & \SI{0.5}{\ul}  & \SI{2.25}{\ul}  & \SI{2.5}{U} \\
	\bottomrule
\end{tabu}
\\

\subsection{PCR Reaktion}

Basierend auf dem Mastermix wurden vier PCR Proben mit jeweils \SI{1.5}{\ul}
Templat-DNA vorbereitet:
\begin{enumerate}
	\item Isolat Fleischextrakt: mtDNA der Fleischprobe
	\item Isolat Extraktionskontrolle
	\item PCR Negativkontrolle: Wasser
	\item PCR Positivkontrolle: Humane mtDNA
\end{enumerate}

PCR-Reaktion auf dem Thermocycler wie folgt durchführen:

\begin{itemize}
	\item 1 Zyklus: Denaturierung, \SI{3}{\min} bei \SI{95}{\celsius}
	\item 40 Zyklen:
		\begin{itemize}
			\item Denaturierung: \SI{15}{\sec} bei \SI{95}{\celsius}
			\item Annealing: \SI{15}{\sec} bei \SI{51}{\celsius}
			\item Extension: \SI{15}{\sec} bei \SI{72}{\celsius}
		\end{itemize}
	\item 1 Zyklus: Letzte Extension, \SI{3}{\min} bei \SI{72}{\celsius}
\end{itemize}

\section{Restriktionsverdau}

\subsection{Mastermix}

Für die drei verwendeten Restriktionsenzyme wurde je ein Mastermix hergestellt:
\\

\begin{tabu}{lll}
	\toprule
	& Volumen 1x & Volumen 2.5x \\
	\midrule
	PCR Produkt & \SI{15}{\ul}  & - \\
	Puffer 10x (spezifisch für Enzym)  & \SI{2.5}{\ul} & \SI{6.25}{\ul} \\
	\ce{H2O}    & \SI{6.5}{\ul} & \SI{16.25}{\ul} \\
	Enzym (AluI \SI{10}{U \per \ul}, HinfI \SI{40}{U \per \ul}, HaeIII \SI{10}{U \per \ul})  & \SI{6.5}{\ul} & \SI{16.25}{\ul} \\
	\bottomrule
\end{tabu}
\\

Daraus wurden sechs Proben angefertigt:
\\

\begin{tabu}{lll}
	\toprule
	Probe & PCR Produkt & Restriktionsenzym \\
	\midrule
	1 & AluI & Gewebeprobe \\
	2 & AluI & Positivkontrolle \\
	3 & HaeIII & Gewebeprobe \\
	4 & HaeIII & Positivkontrolle \\
	5 & HinfI & Gewebeprobe \\
	6 & HinfI & Positivkontrolle \\
	\bottomrule
\end{tabu}
\\

Die Röhrchen wurden für \SI{1}{\hour} bei \SI{37}{\celsius} inkubiert.

\section{Gelelektrophorese}

\subsection{Vorbereitung Agarosegel (\SI{2.2}{\percent} Agarose)}

\begin{itemize}
	\item \SI{1.1}{\g} Agarose in \SI{55}{\ml} 1x TAE-Puffer im
		Mikrowellenofen gut aufkochen.
	\item \SI{3}{\ul} Ethidiumbromidlösung (\SI{10}{\mg \per \ml}) dazu
		pipettieren (Vorsicht mutagen!), mischen und sofort in die
		vorbereitete Gelbox giessen.
	\item Gel erstarren lassen, bis es weisslich trüb ist. Anschliessend in
		die Laufkammer legen und mit 1x TAE-Puffer überschichten.
\end{itemize}

\subsection{Elektrophorese}

\begin{itemize}
	\item Isolierte mtDNA: \SI{5}{\ul} der unverdünnten, isolierten DNA
		werden mit je \SI{5}{\ul} des entsprechen den restlichen
		unbehandelten PCR-Produktes (in den \SI{0.2}{\ml} tubes) in ein
		neues \SI{1.5}{\ml} tube pipettiert und mit weiteren 5 ul
		Ladepuffer versetzt, ergibt \SI{15}{\ul} gesamt, die werden
		komplett geladen.
	\item Verdaute PCR-Produkte: geben Sie direkt zu den \SI{25}{\ul} im
		Restriktionsverdau-Röhrchen \SI{5}{\ul} Ladepuffer dazu, macht
		\SI{30}{\ul} gesamt, die werden komplett geladen.
	\item Pipettieren Sie die Proben nach Ihrem erstellten Schema in die
		Taschen des Gels, wobei Sie den 100 bp Längenstandard
		(\SI{10}{\ul}) nicht vergessen sollten.
	\item Bei \SI{100}{\V} Spannung wird die DNA im elektrischen Feld
		solange aufgetrennt, bis die Farbe des Ladepuffers Orange G
		gerade komplett aus dem unteren Rand des Gels herausläuft.
	\item Das Gel wird im UV-Licht (bei \SI{254}{\nm}) fotografiert.
		(UV-Licht ist sehr gefährlich für die Augen. Es ist deshalb
		unbedingt mit Schutzbrille oder Schutzschild zu arbeiten).
	\item Das Gel wird in den gesonderten Abfall gegeben und der
		Ethidiumbromid-haltige Puffer wird durch Aktivkohle gefiltert.
\end{itemize}

\begin{landscape}

\subsubsection{Ladeschema der Gelelektrophorese}

\begin{tabu}{lp{6.5cm}p{6.5cm}ll}
	\toprule
	Lane & & & Loading Buffer & Ladevolumen \\
	\midrule
	1 & \SI{5}{\ul} PCR Produkt Fleischprobe & \SI{5}{\ul} mtDNA Fleischprobe & \SI{5}{\ul} & \SI{15}{\ul} \\
	2 & \SI{5}{\ul} PCR Produkt Extraktionskontrolle & \SI{5}{\ul} mtDNA Extraktionskontrolle & \SI{5}{\ul} & \SI{15}{\ul} \\
	3 & \SI{5}{\ul} PCR Produkt Negativkontrolle & \SI{5}{\ul} \ce{H2O} & \SI{5}{\ul} & \SI{15}{\ul} \\
	4 & \SI{5}{\ul} PCR Produkt Positivkontrolle & \SI{5}{\ul} \ce{H2O} & \SI{5}{\ul} & \SI{15}{\ul} \\
	5 & \SI{10}{\ul} Size marker & & & \SI{10}{\ul} \\
	6 & & \SI{25}{\ul} Verdau AluI Fleischprobe & \SI{5}{\ul} & \SI{30}{\ul} \\
	7 & & \SI{25}{\ul} Verdau AluI Positivkontrolle & \SI{5}{\ul} & \SI{30}{\ul} \\
	8 & & \SI{25}{\ul} Verdau HaeIII Fleischprobe & \SI{5}{\ul} & \SI{30}{\ul} \\
	9 & & \SI{25}{\ul} Verdau HaeIII Positivkontrolle & \SI{5}{\ul} & \SI{30}{\ul} \\
	10 & & \SI{25}{\ul} Verdau HinfI Fleischprobe & \SI{5}{\ul} & \SI{30}{\ul} \\
	11 & & \SI{25}{\ul} Verdau HinfI Positivkontrolle & \SI{5}{\ul} & \SI{30}{\ul} \\
	12 & \SI{10}{\ul} Size marker & & & \SI{10}{\ul} \\
	\bottomrule
\end{tabu}
\\

Da nur 12 Lanes vorhanden waren, wurden in die ersten vier Lanes zwei zu
analysierende Produkte gegeben. Da sich deren Grösse stark unterscheidet
verursachte dies keine Probleme.

\end{landscape}

\subsection{Resultat der Elektrophorese}

In Abbildung \ref{fig:pcr_elektrophorese} sind die Resultate der
Gelelektrophorese unter UV Licht sichtbar. Die Lanes von links nach rechts
entsprechen dem dokumentierten Ladeschema.

In Lane 1 ist leicht unter dem 400bp Marker eine Bande zu sehen, welche
bestätigt dass das PCR Produkt - dessen erwartete Länge 358bp beträgt -
vorhanden ist. Weit oberhalb des 1000bp Markers ist eine Bande welche das
komplette mitochondrielle Genom ist.

Die Abwesenheit der Banden in Lane 2 bestätigt, dass sauber gearbeitet wurde,
und in der Extraktionskontrolle keine DNA eingeschleust wurde welche durch die
PCR vervielfältigt worden wäre.

Das Gleiche gilt für die PCR Negativkontrolle in Lane 3, wobei dort eine feine
Bande zu sehen ist, was impliziert dass die verwendete Wasserprobe leicht
verschmutzt wurde.

In der PCR Positivkontrolle in Lane 4 bestätigt die Bande bei 358bp dass die
PCR Reaktion erfolgreich war.

Auf die restlichen Lanes wird als Teil der Auswertung eingegangen.

\begin{landscape}

\begin{figure}[h]
	\centering
	\includegraphics[width=0.95\paperwidth]{img/pcr_elektrophorese.png}
	\caption{Resultate der Agarose-Gelelektrophorese, unter UV Licht}
	\label{fig:pcr_elektrophorese}
\end{figure}

\end{landscape}

\chapter{Auswertung}

Alle Verweise auf die verschiedenen Lanes beziehen sich auf Abbildung
\ref{fig:pcr_elektrophorese}.

\section{Bestimmung der Spezies}

Im AluI Verdau der Fleischprobe ist eine doppelte Bande bei circa 200bp, sowie
die originale Bande des unverdauten Segmentes bei 358bp, sichtbar. Dies erlaubt
bereits alle Spezies auszuschliessen bei denen durch AluI nichts verdaut wird,
oder deren AluI Verdau zu Segmenten unpassender Länge führt. Dies sind Giraffe,
Mensch, Schaf, Pute, Huhn und Ziege da dort kein Verdau stattfindet, sowie
Schwein da dort eine Bande bei 114bp erwartet würde.

Im HaeIII Verdau werden jene Banden, die auch beim Menschen - ersichtlich in
der Positivkontrolle - vorkommen ignoriert. Dies verhindert, dass aufgrund
potentieller Verschmutzungen falsche Rückschlüsse gezogen werden. Damit
verbleibt eine Bande bei 300bp sowie eine unter 100bp. Dies erlaubt die
Auschliessung des Pferdes, da dort keine Bande bei 300bp erwartet wird.

Damit bleibt als einzige Möglichkeit das Rind, was mittels des HinfI Verdau
bestätigt werden kann. In jenem Verdau existieren Banden bei 200bp„ leicht über
100bp, und deutlich unter 100bp. Alles dies sind Banden, die beim Rind (198bp,
117bp, 44bp) erwartet werden.

Damit lässt sich mit grosser Wahrscheinlichkeit sagen, dass es sich bei der
Fleischprobe um Rind handelte. Weiterhin ist ersichtlich, dass mindestens der
HaeIII Verdau durch menschliche DNA kontaminiert wurde.

\subsection{PCR-RFLP Muster einiger Tierarten des cytb Systems}

\begin{tabu}{lllllllllll}
	\toprule
	Enzym  & Bison & Giraffe & Mensch & Schaf & Pferd & Pute & Huhn & Rind & Schwein & Ziege \\
	\midrule
	AluI   & 190   & 358     & 358    & 358   & 190   & 358  & 358  & 190  & 244     & 358 \\
	       & 168   &         &        &       & 168   &      &      & 168  & 114     &     \\
	\midrule
	HaeIII & 159   & 285     & 233    & 159   & 159   & 125  & 159  & 285  & 153     & 230 \\
	       & 125   &  73     & 125    & 125   & 125   & 104  & 125  &  73  & 131     &  74 \\
	       &  74   &         &        &  74   &  74   &  74  &  74  &      &  74     &  55 \\
	       &       &         &        &       &       &  55  &      &      &         &     \\
	\midrule
	HinfI  & 267   & 198     & 198    & 198   & 198   & 198  & 187  & 198  & 358     & 198 \\
	       &  91   & 160     & 160    & 160   & 160   & 160  & 161  & 117  &         & 160 \\
	       &       &         &        &       &       &      &  10  &  44  &         &     \\
	       &       &         &        &       &       &      &      &      &         &     \\
	\bottomrule
\end{tabu}

\section{Bestätigung \& Behandlung unklarer Fälle}

Für zusätzliche Sicherheit, oder falls kein klarer Befund möglich wäre, könnte
ein Verdau mit weiteren Restriktionsenzymen durchgeführt werden. Diese würden
zu weiteren Bandenmustern führen, mit welchen abgeglichen werden könnte.
Weiterhin könnten auch andere geeignete Segmentes des Genoms mittels PCR-RFLP
verglichen werden.

Alternativ könnte auch, da inzwischen schnell und günstig machbar, das
komplette oder Teile des Genoms sequenziert, und auf Ähnlichkeit mit bekannten
bereits sequenzierten Genomen untersucht werden.

\section{RFLP von Alec Jeffreys in 1985\cite{Jeffreys1985}}

Alec Jeffreys untersuchte VNTRs in menschlichem Genom - kurze
Nukleotidsequenzen welche an vielen Orten im Genom vorkommen, oft mehrere Male
aneinandergereiht. Die Lage und Länge dieser Repetionen unterscheidet sich
stark zwischen Individuen.

Wird das Genom nun mit Restriktionsenzymen verdaut, und in einer
Gelelektrophorese der Grösse nach angeordnet, können radioaktiv markierte
Marker zugegeben werden welche diese VNTRs erkennen und binden. Auf einem Film
kann nun die Position und Anzahl dieser VNTRs aufgrund des radioaktiven
Zerfalls der gebundenen Marker visualisiert werden, was zu einem individuellen
genetischen Fingerabdruck führt.

\section{Vollständigkeit des Restriktionsverdaus}

Um zu zeigen ob der Restriktionsverdau vollständig war, müssen die Lanes aller
Verdaue mit der Lane, welche das unverdaute Segment enthält, verglichen werden.

Relevant sind alle Verdaue, bei denen verdaute Segmente sichtbar sind. Ist bei
allen diesen Verdauen keine Bande auf Höhe des unverdauten Segments ersichtlich,
war der Verdau komplett. Hat hingegen mindestens ein Verdau zwei Bande, wobei
eine auf Höhe des unverdauten Segmentes ist, war der Verdau nicht komplett.

In unserem Fall ist klar, dass der Verdau nicht komplett war. So enthält
beispielsweise der HinfI Verdau sowohl die Bande des vollen Segmentes, als auch
Produkte des Verdaus.

\section{Qualität der isolierten DNA}

Aus der Nanodrop Spektroskopie - spezifisch dem 260:280 Verhältnis - ist
ersichtlich, dass die isolierte DNA nur leicht mit Proteinen verunreinigt war.
Weiterhin liefert die Spektroskopie eine Konzentration der DNA im
aufgereinigten Sample.

In der Gelelektrophorese ist sichtbar, dass zumindest der HaeIII Verdau mit
humaner DNA kontaminiert wurde. Da diese Kontamination aber im HinfI Verdau
nicht sichtbar ist, wird nicht die isolierte mtDNA sondern potentiell die Probe
des Verdaus kontaminiert worden sein.

\section{Kritischste PCR Temperatur}

Die kritischste Temperatur der PCR ist jene des Annealing. Ist sie zu hoch, so
kann sich der Primer nicht anlagern, und das Segment wird nur ungenügend oder
gar nicht repliziert. Ist sie zu tief so sind die Primer nicht spezifisch
genug, und es werden jenste alternative Segmente repliziert.

\section{Genetische Voraussetzungen für PCR-RFLP}

Um PCR-RFLP anwenden zu können, müssen zwei Voraussetzungen gegeben sein. Zum
einen müssen zwei Stellen im Genom zwischen den zu unterscheidenden Spezies
konserviert sein, so dass die Primer sich bei Proben aller Spezies daran
anlagern können. Zum zweiten muss die Region zwischen den Primern variabel sein
- das heisst es muss genügend Unterschiede geben, dass der Verdau durch
verschiedene Restriktionsenzyme aussagekräftig ist.

\bibliographystyle{vancouver}
\bibliography{references}

\end{document}
